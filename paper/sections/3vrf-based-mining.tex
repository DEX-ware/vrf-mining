\section{VRF-based mining}

VRF-based mining is simple and intuitive: it just replaces the hash function with a VRF.
Instead of using a hash function, VRF-based mining uses a VRF to produce hashes that satisfy the difficulty requirement.
Different from hash functions, VRF takes both an input string and a secret key to produce a hash.
In addition, the owner of this secret key can produce a proof which proves that the hash is generated by himself.
Thus, to outsource the mining process, the pool operator should show his secret key to miners.
However, the pool operator should keep his private key confidential, otherwise a miner in his mining pool can steal all his coins.

\RH{We should align to Scratch-off puzzle paper's Definition 1. More specifically,
1. symbols
2. we should not call these two phases miner side, node side, phase 1 and phase 2. Instead, the first algorithm is Work, the second is Prove (we have but Miller's paper does not have), and Verify.
Runchao will do this}


In the case of solo mining, our scheme, which contains a miner side and a node side, works as follows. Note that we divide the blockheader into two part:
one part containing $pk$, $\beta$, and $\pi$ (which will be explained later);
and the other part not containing them.
We call the later part $\alpha$.

\RH{revise this pseudocode according to Vincent Gramoli's paper. In particular, we should add inline comments. Runchao will do this}


\textbf{Work.}
A Miner performs mining as described in Algorithm ~\ref{algo:work}.

\begin{algorithm}[H]
\caption{Work.}
\label{algo:work}
\SetAlgoLined
  $(sk, pk) \gets \mathsf{VRFKeyGen}(1^{\lambda})$\;
  build a block template, including a coinbase transaction paying to the \texttt{scriptPubKey} of $pk$\;
  $\beta \gets \mathsf{VRFHash}(sk, \alpha)$\;
  \While{$\beta > target$}{
    alter the nonce in $\alpha$\;
    $\beta \gets \mathsf{VRFHash}(sk, \alpha)$\;
  }
  $\pi \gets \mathsf{VRFProve}(sk, \alpha)$\;
  submit $pk$, $\beta$, and the proof $\pi$ to the node.
\end{algorithm}






\textbf{Prove.}
Upon receiving $pk$, $\beta$, and $\pi$, the node then performs Algorithm~\ref{algo:vft-mining-2}.

\begin{algorithm}[H]
\caption{VRF Mining Phase 2}
\label{algo:vft-mining-2}
\SetAlgoLined
\If{$\beta$ $\le$ target}{
    \If{$\mathsf{VRFVerify}(pk, \alpha, \beta, \pi)$}{
        \If{the coinbase transaction is paying to the \texttt{scriptPubKey} of $pk$}{
        record $pk$, $\beta$ and $\pi$ in the block\;
        produce a new block, with coinbase reward paying to the miner\;
        propogate the block to other nodes\;
        }
    }
}
\end{algorithm}

The $pk$, $\beta$ and $\pi$ are recorded in the blockheader, so that, when the block is propogated to other nodes, they can easily validate it via $\mathsf{VRFVerify}(pk, \alpha, \beta, \pi)$.




\textbf{Verify.}

\TODO{write verify}