\section{VRF-based mining}

\subsection{Our construction}

We replace hash function in mining by VRF, and thus construct VRF-based mining.

To rule out pooled mining, we combine VRF with digital signature scheme.
As the digital signature scheme use in cryptocurrency are based on Elliptic curve, we construct Elliptic curve based VRF for our VRF-based mining.
The standardised construction is put in the appendix~\ref{vrf_standardised_construction}.

% The desired properties of a well-designed hash function includes uniformity, deterministic, unpredictability and verifiability.
% As shown in appendix~\ref{vrf_standardised_construction}, ...
% Therefore, VRF fits in here.

In the case of solo mining, our sheme works as follows.

\begin{enumerate}
    \item the solo miner runs $\mathsf{VRFKeyGen}$ to generate a secret key and public key pair;
    \item ...
    \item ...
    \item miner then submit the $pk$, $\beta$, and the proof $\pi$ to the full node it runs;
    \item full node then check against $\mathsf{VRFVerify}(pk, \alpha, \beta, \pi)$, if the result is valid, a new block is mined, and the coinbase reward is paid to the miner, for example, to his scriptPubKey (thus, the miner can spend the reward by his private key later);
    \item $pk$, $\beta$ and $\pi$ should also be recorded in the block header, so that other nodes can easily validate via $\mathsf{VRFVerify}(pk, \alpha, \beta, \pi)$.
\end{enumerate}

\subsection{Why mining pools cannot work in our scheme}

% need private key
As shown in appendix~\ref{vrf_standardised_construction}, both $\mathsf{VRFHash}$ and $\mathsf{VRFProve}$ take secret key $sk$ as input.

That is to say, if a pool operator would like to provide pooled mining service and let miners participate, he will need to provide miners with his scret key.

% no client would like to provide private key

% balance be spent