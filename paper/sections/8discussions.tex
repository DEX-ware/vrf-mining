\section{Discussions}
\label{sec:discussions}

While eliminating mining pools can improve decentralisation, it also introduces new problems.
We discuss two problems and their possible solutions, namely 1) high reward variance and 2) secret key leakage in memory.

\subsection{Weaker security guarantee of PoW-based consensus}

The reason of miners joining mining pools is that, miners may not obtain stable income via solo mining.
Miners may be discouraged to mine if their rewards are highly volatile.
With weaker incentive of mining, fewer miners will join the blockchain.
The blockchain will then attract less mining power, which weakens the security of PoW-based consensus.

This problem can be addressed by making the mining reward fine-grained.
% multi-tier
Miller et al.~\cite{miller2015nonoutsourceable} proposes the \emph{multi-tier reward scheme}.
In multi-tier reward scheme, the mining difficulty is divided into different levels, and miners can mine blocks satisfying different difficulty levels arbitrarily.
In this way, mining reward is distributed in a fine-grained way, so lowers the reward variance.
% strongchain
StrongChain~\cite{szalachowski2019strongchain} introduces the notion of \emph{Collaborative PoW}, where miners are encouraged to mine blocks together and the mining reward is distributed in proportion to miners' contributions.
Bobtail~\cite{bissias2020bobtail} and HotPoW~\cite{keller2019hotpow} further explore the Collaborative PoW approach.

Another solution is to increase the rate of mining blocks.
With more blocks mined in a time unit, the mining reward can also be more stable.
For example, although of independent interest, protocols for scaling blockchains such as sharding~\cite{wang2019monoxide} and DAG~\cite{li2018scaling} can also stabilise the mining reward variance.



\subsection{Secret key leakage in memory}

VRF-mining requires the secret key, so a miner should keep its secret key in plaintext in memory \emph{all the time}.
This gives adversaries opportunity to steal the secret key from the memory.
For example, if the mining software has a bug that enables hackers to access the memory space of the mining software, then the hacker can easily steal the secret key.

% why inevitable
For VRF-based mining, keeping secret keys in memory is inevitable.
To achieve \emph{non-outsourceability}, the secret key should be combined to the mining process.
As long as mining does not stop, the secret key should be kept in plaintext in memory.
This applies to all protocols that execute frequently and require secret keys, such as TLS.

There have been several different ways to protect in-memory secret keys.
% enclave
First, the miner can isolate the scalar multiplication to a software or hardware enclave.
Note that the encalves are unnecessary to be general-purpose.
% proactive self-destroy
Second, the process of the mining software can destroy itself once detecting anomalous memory access attempts.
% isolation
Third, the miner can isolate mining from the blockchain node.
For example, it can run the mining process independently with its blockchain node process.
In this way, if an adversary compromises its node but not the operating system, the adversary still cannot read the private key in another process.
% offline
Fourth, it can run the mining process in an offline machine that can only communicate with its blockchain node.
% oram
Fifth, Oblivious RAM (ORAM) is a promising primitive to protect sensitive data in memory.
ORAM allows CPUs to access data in memory while the data in memory is encrypted and the access pattern is hidden.
% minimise
Last, the miner can use a new secret key for each block, so that leaking a secret key only affects the reward of a single block.
