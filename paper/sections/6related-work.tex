\section{Related work}

We investigate several proposals aiming at increasing the decentralisation level of Bitcoin, and find that none of them completely discentivise mining pools.
Our analysis also show that some of them suffer from significant flaws.

\paragraph{\textbf{Nonoutsourceable Puzzles}}
Miller et al.~\cite{miller2015nonoutsourceable} formalize the definition of \textit{Nonoutsourceable Puzzles},
\TODO{tiers}
\TODO{and construct a \textit{Weekly nonoutsourceable puzzles}}
\TODO{flaws}
Moreover, our protocol in fact construct a stronger version of their \textit{Strongly nonoutsourceable puzzles} if integrated with zero-knowledge transformation as in their protocol, because if the pool manager outsources the computation task, he will not only lose the block rewards, but also the secret key, result in losing all his balance, identity forging and privacy leakage.

% P2Pool
\paragraph{\textbf{P2Pool}}
successful
low variant
but: efficiency, security, incentive



\paragraph{\textbf{SmartPool}}
% SmartPool: Practical Decentralized Pooled Mining
% https://www.usenix.org/system/files/conference/usenixsecurity17/sec17-luu.pdf

\paragraph{\textbf{Adaptive Retaliation Strategies}}
Kwon et al.~\cite{kwon2019eye} proves that in a Repeated FAW\cite{courtois2014subversive,rosenfeld2011analysis}-BWH\cite{kwon2017selfish} Game, a larger pool can make profit whereas a smaller pool suffers a loss, which will undoubtedly increase the centralisation level of Bitcoin.
They then bring up \textit{Adaptive Retaliation Strategies} (\textit{ARS}) to make FAW attack and BWH attack unprofitable, so as to discentivise rational pools in conducting such attacks.
However, \textit{ARS} increases the operation cost for a pool manager, because he needs to plant moles to other pools, calculate on the models and detect opponents' attacks.
Moreover, in \textit{ARS}, a large pool with enough budget can still perform sabotage attack to drain small pools fund and force them to quit the market, leveling up the centralisation.

\paragraph{\textbf{2P-PoW}}
Eyal and Sirer~\cite{2P-PoW} propose \textit{Two Phase Proof of Work} (\textit{2P-PoW}) for smooth transition to large pool disincentivisation, by dividing PoW into phase 1 as same as the original minig and phase 2 requiring a secret key.
This solution is backward-compatible because it preserves blockchain data and retains mining hardware. 
However, \textit{2P-PoW} still suffers from several problems:
\begin{itemize}
\renewcommand\labelitemi{$\bullet$}
    \item it does not deter mining pools but merely reduce centralisation, since phase 1 is outsourcable;
    \item \sout{$\mathsf{SIG}$ in phase 2 is usually achieved by \textit{ECDSA} or \textit{EdDSA}, containing a changable cryptography parameter, which leads to another traversal space (such a problem does exist in VRF because the parameter is determined once a VRF is chosen);} \HY{deleted because, in Eyal et al.'s post, they state that a deterministic $SIG$ is needed}
    \item how to choose the difficulty parameters for each phase still remains unknown and needs simulation;
    \item it still requires a hard fork.
\end{itemize}



% BetterHash
% \paragraph{\textbf{BetterHash}}
% BetterHash~\cite{draft-bip-BetterHash} is one of the solutions to the mining power centralisation.
% However, BetterHash suffers from criticism because:
% \begin{itemize}
% \renewcommand\labelitemi{$\bullet$}
%     \item it does not solve the centralised pooled mining, but only transfers the right to build the block template to the miner side;
%     \item it requires miner technical knowledge to create a block templeate, manage and run a full node.
% \end{itemize}
