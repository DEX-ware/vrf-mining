\section{Related work}
\label{sec:related}

To the best of our knowledge, VRF-based mining is the first construction that makes pooled mining \textit{impossible}.
We briefly review related research on preventing mining pools, and compare them with VRF-based mining.
We classify related research to two types, namely pooled-mining-unfriendly mining protocols and decentralised mining pools.

\begin{table}[t]
    \centering
    \caption{Comparison between mining protocols. NSP is short for Non-outsourceable scratch-off puzzle.}
    \begin{tabular}{ccccc}
        \hline
                                        & VRF-based mining & NSP-1 & NSP-2 & 2P-PoW \\ \hline
        Punishment                    & New reward           & New reward                                 & New reward                                 & New reward               \\
        Stealing                      & Unlinkable       & Linkable                                   & Unlinkable                                 & Unlinkable           \\
        No partial outsourcing        & \cmark           & \cmark                                     & \cmark                                     & \xmark               \\
        Support randomised signatures & \cmark           & \cmark                                     & \cmark                                     & \xmark               \\
        No complex cryptography       & \cmark           & \cmark                                     & \xmark                                     & \cmark               \\ \hline
    \end{tabular}
    \label{table:comparison-mining-protocols}
\end{table}

\begin{table}[t]
    \centering
    \caption{Comparison with decentralised mining pools.}
    \begin{tabular}{ccccc}
        \hline
                            & VRF-based mining & P2Pool & SmartPool & BetterHash \\ \hline
        Complexity       & -                & Blockchain                     & Smart contract                    & -                                      \\
        Decentralisation & Mining           & Mining                         & Mining                            & Select txs                             \\ \hline
    \end{tabular}
    \label{table:comparison-pools}
\end{table}



\subsection{Mining protocols}

There are two mining protocols aiming at discouraging or breaking mining pools: the \textit{non-outsourceable scratch-off puzzle}~\cite{miller2015nonoutsourceable} and the \textit{Two Phase Proof-of-Work} (\textit{2P-PoW})~\cite{2P-PoW}.
Table~\ref{table:comparison-mining-protocols} summarises our comparisons.

\textbf{Non-outsourceable scratch-off puzzle.}
Miller et al.~\cite{miller2015nonoutsourceable} formalises mining as \emph{scratch-off puzzles}, defines \emph{non-outsourceability}, and proposes two \emph{non-outsourceable scratch-off puzzles}.
One achieves \emph{weak non-outsourceability} --- miners can steal the mining reward, and the other achieves \emph{strong non-outsourceability} --- miners can steal the mining reward anonymously.

The weak non-outsourceable scratch-off puzzle works as follows.
First, the miner randomly generates a Merkle tree.
Second, the miner randomly chooses a nonce, samples some leaves of the Merkle tree according to the nonce, and hashes their Merkle proofs together to a single hash.
If this hash meets the difficulty requirement, then the nonce is valid.
Last, the miner binds the valid nonce and his block template together to a valid block.

In order to outsource the mining process, the mining pool should distribute the search space of nonces to miners.
% how to steal
A miner can steal the mining reward by binding valid nonces it finds with his own block template.
% why weak
However, this stealing process is not anonymous.
Unlike existing PoW-based consensus where miners can choose both nonces and block templates, the weak non-outsourceable scratch-off puzzle only allows miners to choose nonces, so all miners share the same search space of nonces.
Thus, the pool operator can link the nonce in the stolen block with the miner who is assigned with the search space covering this nonce.
To achieve \emph{strong non-outsourceability}, the strong non-outsourceable scratch-off puzzle replaces the plaintext nonce in the block with a Zero Knowledge Proof proving the statement ``I know a valid nonce''.

As discussed in \S\ref{sec:non_outsourceability}, while being simpler and more efficient than non-outsourceable scratch-off puzzle, our VRF-based mining achieves \emph{strong non-outsourceability}.

\textbf{2P-PoW.}
Eyal and Sirer proposes 2P-PoW~\cite{2P-PoW}, a mining protocol that discourages pooled mining.
In 2P-PoW, mining consists of two phases, and each phase has a difficulty parameter.
A miner should find a nonce that makes the block to pass two phases: 1) the SHA256D hash of the block meets the first difficulty, 2) the SHA256 hash of the signature of the block meets the second difficulty.
As the second requirement requires the private key, pool operators cannot outsource the second phase.

% first phase outsourceable
Compared to VRF-based mining that makes pooled mining impossible, 2P-PoW is partially outsourceable by outsourcing the first phase.
% det sig
In addition, 2P-PoW requires the digital signature algorithm to be deterministic, while commonly used digital signatures such as ECDSA rely on randomisation.
If the signature is randomised, the pool operator can make use of all nonces from miners that meet the first phase but fail the second phase.
Given a nonce meeting the first difficulty, the pool operator repetitively generates signatures to meet the second requirement.
% diff
Moreover, how to adjust two difficulties still remains unknown and requires some simulations.




\subsection{Decentralised mining pools}


% ref: https://www.alexeizamyatin.me/files/Decentralized_Mining-Security_and_Attacks.pdf

Decentralised mining pool is a type of mining pools that work in a decentralised way: miners mine in the name of themselves and share reward in a fine-grained way.
In this way, miners are rewarded stably while mining power is not controlled by pool operators.
Table~\ref{table:comparison-pools} summarises the comparison between VRF-based mining with decentralised mining pools.


\textbf{P2Pool}~\cite{voight2011p2pool} is a decentralised mining pool for Bitcoin.
% how p2pool works
Instead of using a centralised server, P2Pool uses a blockchain called \emph{share-chain} to receive shares from miners.
All miners in P2Pool participate in the PoW-based consensus of share-chain.
Once a miner finds a share on Bitcoin, he appends a block to the share-chain.
The block consists of the share and a coinbase transaction that rewards miners in proportion to their shares.
Once a miner mines a block that also satisfies Bitcoin's difficulty, he submits this block to the Bitcoin blockchain, and and miners are rewarded according to the coinbase transaction.
% challenges
P2Pool suffers from several limitations.
First, handling the difficulty of mining shares is hard.
If the difficulty is high, miners' reward will still be volatile.
If the difficulty is low, there will be numerous low-difficulty shares, which introduces huge overhead on broadcasting and receiving shares.
Second, shares on share-chain are much more frequent than blocks on Bitcoin blockchain, and the share-chain suffers from high orphan rate.
Last, existing research shows that P2Pool is vulnerable to temporary dishonest majority~\cite{decentralised-mining-pool-security}.

\textbf{SmartPool}~\cite{luu2017smartpool} is another decentralised mining pool.
Instead of using a blockchain, SmartPool employs smart contracts to track shares from miners.
% con 1: smart contract
This implies that SmartPool cannot work on blockchains without smart contracts.
% con 2: network delay
In addition, as blockchains achieve limited throughput, blockchains can only handle a limited number of shares for each time unit.
In this way, distributing mining reward may also be delayed, especially when a large number of miners participate in the SmartPool.
% con 3: compute and storage overhead
Moreover, as the SmartPool smart contract should verify the validity of blocks, miners should submit the entire block --- including transactions --- to the SmartPool.
Verifying blocks introduces non-negligible computing overhead and transaction fee, and storing blocks in smart contracts also introduces non-negligible overhead on storage.



\textbf{BetterHash}~\cite{draft-bip-BetterHash} is another decentralised mining protocol, which has been integrated into \emph{Stratum V2}~\cite{stratum-v2}, the next generation of the \textit{Stratum}~\cite{stratum} pooled mining protocol.
In \textit{BetterHash}, the block operator allows miners to choose transactions and construct blocks in his name, rather than constructing block templates by himself.
Thus, \textit{BetterHash} only contributes to the decentralisation of constructing blocks, but does not improve the decentralisation of mining power.