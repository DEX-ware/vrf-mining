\section{Introduction}

% Bitcoin
By December 2019, Bitcoin is the most popular cryptocurrency in the world.
It uses a public ledger to record the transaction history: transactions are stored in blocks, and each block is linked to its previous block.
% TODO: PoW
% mining
In Bitcoin, Miner search for a PoW solution to a puzzle, and the miner outrun others become the leader to produce the next block and win the reward.
Though in Proof of Stake~\cite{} and its variants, there is also mining mechanism that, coins are minted and rewarded to the miner for each block, we only discuss mining in Proof of Work in this paper.
\TODO{we can also discuss dPoS?}

% TODO: mining pool

% TODO: by luck, short term, long term
Payouts will be less frequent for a solo miner controlling low computing power. Miner need to pay for electricity periodically and get the investment back.
Inconsistent payouts will be risky for miners.
To reduce the risk, miners join mining pool.
no need to run full node and deal with block creation, block propogation
can focus on running mining rigs, for example, ASICs.

Pool operator, run full node, distribute/outsource task
produce a block if find solution meeting target difficulty.

However, mining in Bitcoin network suffers from centralisation.

% show centralisation

\RH{describe mining pool breaks the decentralisation}


1. malicious attack
double spend, 51\%, with selfish mining, only requires ... 33?25?
1.1. a group of pool operators can collude.
1.2. even not collude, the cost is not huge, lower pool fee

2. Even not for 51, 
pool operators may refuse protocol upgrade. \TODO{for what? why bad?}


3. DNS, steal miners power


In this work, we put forward a novel idea of combining VRF and mining in PoW to eliminate the intentionality of pooled mining, and to conquer the centralisation of mining power. We call our constrcution \textbf{VRF-based mining}.