\section{Introduction}
\label{sec:intro}

Bitcoin~\cite{nakamoto2008bitcoin} started the era of cryptocurrency.
% blockchain
In Bitcoin, each participant maintains its own ledger.
The ledger is structured as a blockchain, i.e., a chain of blocks of transactions.
% proof of work
Participants keep executing Proof-of-Work (PoW)-based consensus to agree on blocks and append them to the blockchain.
In PoW-based consensus, each participant keeps trying to solve a computationally hard PoW puzzle~\cite{dwork1992pricing}.
Once solving the puzzle, the participant appends its block to the blockchain and gets some coins as reward.
By adaptively adjusting the difficulty of PoW puzzles, PoW-based consensus can ensure only one participant solves the puzzle for each time period with high probability.
Participants of PoW-based consensus are also known as miners, the process of solving PoW puzzles is known as mining, and the computation power used for mining is known as mining power.

% why pooled mining
Due to the limited rate of producing blocks, a miner's reward can be highly volatile if it chooses to mine by himself.
To stabilise mining rewards, miners usually choose to join mining pools.
A mining pool allows miners to mine on blockchain in the name of the pool operator, and the pool operator distributes the mining reward to solo miners in a fine-grained way.
% Problems of having mining pools
While miners are willing to participate in mining pools, mining pools lead to mining power centralisation.
Nowadays, four largest mining pools control more than 51\% Bitcoin mining power~\cite{btc-com}, which can be a lurking threat to blockchain ecosystem.
However, mining power centralisation contradicts Bitcoin's design objective -- decentralisation, and weakens PoW-based consensus' security~\cite{nakamoto2008bitcoin, eyal2018majority}.
% existing solutions
To avoid this, researchers have been working on re-decentralising mining power, mainly by non-outsourceable mining~\cite{miller2015nonoutsourceable,2P-PoW} and decentralised mining pools~\cite{voight2011p2pool, luu2017smartpool, draft-bip-BetterHash}.
However, existing approaches are either inefficient or ineffective, which will be discussed in \S\ref{sec:related} later.

This paper introduces \textit{VRF-based mining}, a surprisingly simple but effective approach to make pooled mining impossible.
VRF-based mining replaces hash functions in PoW-based consensus with Verifiable Random Functions (VRFs)~\cite{micali1999verifiable}.
VRF requires a miner to input its secret key to compute the output, so a pool operator should reveal its secret key to miners in order to outsource mining.
Thus, any miner can steal mining reward of a block owned by the pool operator anonymously.
Our contributions are as follows.

\textbf{We analyse the non-outsourceability of VRF-based mining.}
Miller et al.~\cite{miller2015nonoutsourceable} first formalises the \emph{non-outsourceability} of mining in PoW-based consensus.
We find \emph{non-outsourceability} consists of two orthogonal properties, namely the punishment on pool operators and the unaccountability of punishing.
We then informally prove VRF-based mining achieves the strongest guarantee on both properties.

\textbf{We discuss considerations for instantiating VRF-based mining.}
VRF has four customisable components, namely the elliptic curve, two hash functions between strings and elliptic curve elements, and a normal hash function.
We discuss considerations on choosing them for VRF-based mining.

\textbf{We evaluate feasibility of VRF-based mining.}
We implement Elliptic curve-based VRF specified in~\cite{goldberg2017draft} using Go programming language, and compare its performance with three mainstream mining algorithms SHA256D, Scrypt and CryptoNight.
Our results show that VRF-based mining is easy to implement and introduces negligible overhead compared to hash-based mining.

\textbf{We show that partial outsourcing is unrealistic.}
Partial outsourcing is that, a pool operator interactively outsources computation that does not need the secret key to miners.
We show that partial outsourcing is unprofitable, as it's both computation-intensive and I/O-intensive.

This paper is organised as follows.
Section~\ref{sec:preliminaries} and Section~\ref{sec:construction} describes preliminaries and the construction of VRF-based mining, respectively.
Section~\ref{sec:non_outsourceability} revisits the definition of non-outsourceability, and shows VRF-based mining achieves better non-outsourceability than existing proposals.
Section~\ref{sec:instantiation} discusses concerns of instantiating VRFs for VRF-based mining.
Section~\ref{sec:practicality} provides an experimental analysis on the practicality of VRF-based mining.
Section~\ref{sec:partial-outsourcing} discusses why partial outsourcing in VRF-based mining is unprofitable.
Section~\ref{sec:discussions} discusses potential problems in VRF-based mining and their solutions.
Section~\ref{sec:related} summaries related work, and Section~\ref{sec:conclusion} concludes this paper.
We attach the pseudocode of the standardised EC-VRF construction~\cite{goldberg2017draft} in Appendix.
