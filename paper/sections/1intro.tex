\section{Introduction}
\label{sec:intro}

Bitcoin~\cite{nakamoto2008bitcoin} started the era of cryptocurrency.
Bitcoin's main novelty is Nakamoto consensus - the first consensus protocol that can work in permissionless settings.
% blockchain
In Nakamoto consensus, each participant maintains its own ledger, and the ledger is usually called blockchain.
A blockchain is formed as a chain of blocks, and each block contains a batch of transactions.
Blockchain is designed to be append-only, and its integrity is protected by digital signatures.
% proof of work
Participants keep appending blocks to the blockchain.
To append a block, a participant should first solve a Proof-of-Work (PoW)~\cite{dwork1992pricing} puzzle - a probabilistic and moderately hard computation puzzle.
Once solving the puzzle, the participant will append his block to the blockchain, and get some cryptocurrency as reward.
By adaptively adjusting the difficulty of PoW puzzles, Nakamoto consensus can stabilise the speed of generating blocks.
Participants of Nakamoto consensus are also known as miners, the process of solving PoW puzzles is known as mining, and the computation power used for mining is known as mining power.

% why pooled mining
Today, mining pools dominate the mining power of most blockchains using Nakamoto consensus.
Mining pool is a kind of service that gathers mining power from solo miners and rewards solo miners in a more fine-grained way.
More specifically, a single blockchain may consist of numerous miners, and each miner may have a low chance to mine the next block, leading to unstable mining reward.
A mining pool allows miners to mine on blockchain in the name of the pool operator, and the pool operator distributes the mining reward to solo miners according to their contributed mining power.
In this way, a miner can get reward more stably by joining a mining pool.

% Problems of having mining pools
However, mining pools lead to centralisation of mining power, which contradicts the design objective of Bitcoin - the decentralisation.
% problem of centralisation
Centralisation of mining power can be harmful for the security of Nakamoto consensus.
A mining pool with sufficient mining power can perform numerous types of attacks to break the consensus, such as selfish mining attacks~\cite{eyal2018majority} and 51\% attacks~\cite{nakamoto2008bitcoin}.
Even top mining pools are not big enough, multiple mining pools can even collude to launch attacks.
In addition, mining pools can decide which transactions to include in order to censor the blockchain.
% to date...
To date (01/12/2019), four largest mining pools control more than 51\% Bitcoin mining power~\cite{btc-com}, which can be a lurking threat of Bitcoin.




\textbf{Our contributions.}
In this work, we introduce \textit{VRF-based mining}, a novel idea of using Verifiable Random Functions (VRFs) rather than hash functions for mining in PoW-based consensus.
With VRF, a miner should use his private key to mine blocks, so a pool operator should reveal his private key in order to outsource the mining process to miners.
In this way, no one can outsource the mining process, otherwise any miner can steal all cryptocurrency in his wallet anonymously.

Our contributions are as follows:

\begin{description}
    \item [VRF-based mining] We propose the idea of VRF-based mining, and describe its detailed construction. The construction is surprisingly straightforward: we replace the hash function in PoW-based consensus with a VRF, and a block in the blockchain should contain the VRF hash and its proof, but does not need to contain the digital signature. By verifying the VRF hash, one can convince the correctness of the hash, the satisfiability of the nonce, and the authorship of the block in the same time.
    \item [Revisiting non-outsourceable cryptocurrency mining] We revisit the definition of non-outsourceability of Miller et al.~\cite{miller2015nonoutsourceable}. In particular, we show non-outsourceability consists of the punishment of outsourcing and the unaccountability of breaking pooled mining. We then show how VRF-based mining achieves stronger punishment of outsourcing than non-outsourceable scratch-off puzzle of Miller et al.~\cite{miller2015nonoutsourceable}.
    \item [Instantiating VRFs for VRF-based mining] We discuss how to instantiate VRFs for VRF-based mining. In particular, we show a VRF has four tweakable components, namely the elliptic curve and two hash functions between strings and elliptic curve elements, and a normal hash function. Then, we discuss concerns how to choose them for VRF-based mining purpose.
    \item [Problems and solutions of having no mining pools] We discuss two problems of having no mining pools, namely the high mining reward variance and the high orphan block rate. We then discuss how to address them without mining pools.
\end{description}

\textbf{Paper organisation.}
\S\ref{sec:preliminaries} and \S\ref{sec:construction} describes preliminaries and the construction of VRF-based mining, respectively.
\S\ref{sec:non_outsourceability} revisits the definition of non-outsourceability, and shows VRF-based mining achieves better non-outsourceability than existing proposals.
\S\ref{sec:instantiation} discusses concerns of instantiating VRFs for VRF-based mining.
\S\ref{sec:problems-and-solutions} discusses problems of having no mining pools and their solutions.
\S\ref{sec:related} summaries related work, and \S\ref{sec:conclusion} concludes this paper.
