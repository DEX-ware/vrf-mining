\section{Introduction}

% Bitcoin
By December 2019, Bitcoin is the most popular cryptocurrency in the world.
It uses a public ledger to record the transaction history: transactions are stored in blocks, and each block is linked to its previous block.
In Bitcoin, Proof of Work (PoW) is used to timestamp transactions, and to determine the majority agreement on the transaction history.
% mining
Miners search for a PoW solution to a puzzle concurrently, and the miner who outruns others becomes the leader to produce the next block and win the reward.
Such a reward is to incentivise the miners to maintain the ledger and maintain it honestly.
Though in Proof of Stake~\cite{} and its variants, it is similar that coins are minted and rewarded to the miner for each block, it requires no hash calculation to perform puzzle solution searching.
Therefore, we only discuss mining in PoW in this paper.

Because of the hash results are distributed uniformly and cannot be predicted, the probabilty of a miner to find a solution depends on the hash power it ownes.
Therefore, payouts will be less frequent for a solo miner controlling low computing power.
As miners need to pay for electricity periodically and want to estimate the remaining time to get the investment on hardware back (if they have not), inconsistent payouts can be risky for miners.

% mining pool
To reduce the risk, miners join mining pool to complete the computation tasks.
This also saves miners' effort on running full node and dealing with block creation, block propogation. 
Miners can then focus on running mining rigs, for example, ASICs.

Pool operator is then responsible to run full node, distribute the computing task to miners, 
produce a block if the submitted solution found meeting work requirement.
If the solution does not meet the block work requirement but a reduced requirement, which is called ``pool difficulty'', the miner is also paid, depending on its mining power, because it helps eliminate the incorrect answer.
Thus, mining pool can gather considerable amount of computation power, gain higher probability to mine a block and pay to miners consistently.

% centralisation
However, mining in Bitcoin network suffers from centralisation.


\RH{describe mining pool breaks the decentralisation}


1. malicious attack
double spend, 51\%, with selfish mining, only requires ... 33?25?
1.1. a group of pool operators can collude.
1.2. even not collude, the cost is not huge, lower pool fee

2. Even not for 51, 
pool operators may refuse protocol upgrade. \TODO{for what? why bad?}


3. DNS, steal miners power


In this work, we put forward a novel idea of combining VRF and mining in PoW to eliminate the intentionality of pooled mining, and to conquer the centralisation of mining power. We call our constrcution \textbf{VRF-based mining}.