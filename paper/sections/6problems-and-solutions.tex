\section{Possible problems and solutions}
\label{sec:problems-and-solutions}

While eliminating mining pools can contribute to decentralisation, it will also introduce new problems.
In this section, we discuss potential problems introduced by VRF-based mining, as well as how to address them.
In particular, we focus on two problems, namely the high reward variance and the high orphan rate.

\subsection{High reward variance}

As aforementioned, mining is probabilistic, and miners may not obtain stable income via solo mining, leading to the need of pooled mining.
In other words, unstable income may discourage miners to mine.
With weaker incentive of mining, the blockchain will have less mining power, which weakens the security of PoW-based consensus.
In particular, adversaries can make use of external mining power to compromise blockchains with less mining power~\cite{hansucker}.

A promising approach to address this problem is fine-graining the mining reward scheme.
% multi-tier
Miller et al.~\cite{miller2015nonoutsourceable} proposes the Multi-tier reward scheme, where the mining difficulty is divided into different levels, and miners can mine blocks satisfying different difficulty levels arbitrarily.
The multi-tier reward scheme distributes the mining reward in a fine-grained way so lowers the reward variance.
% strongchain
StrongChain~\cite{szalachowski2019strongchain} introduces the notion of Collaborative PoW, where miners are incentivised to mine blocks together and the mining reward is distributed in proportion to miners' contributions.
% prism
Prism~\cite{bagaria2019prism} decouples the block to core blocks and transaction blocks.
Core blocks construct the blockchain and only contain metadata, while transaction blocks are anchored on core blocks and contain confirmed transactions.
Mining core blocks and transaction blocks is different in terms of difficulty, and miners can mine core blocks and transaction blocks simultaneously.
In this way, miners are also rewarded more stably.

Another approach is to increase the speed of mining blocks.
With more blocks mined in a time unit, the mining reward can also be more stable.
For example, proposals on blockchain scalability such as sharding~\cite{wang2019monoxide} and DAG~\cite{li2018scaling} can stabilise the mining reward variance, although they are not designed for it.





\subsection{High orphan block rate}

Another problem is the high orphan block rate.
Due to the network latency, miners may mine blocks on the same height, but the blockchain will only accept one block and discard the rest blocks at last.
Such discarded blocks are called orphan blocks, and orphan blocks cannot obtain any mining reward.
As miners cannot mine together with VRF-mining, there will be more miners in the blockchain, leading to more orphan blocks.
With higher orphan block rate, miners will waste more mining power and obtain less mining reward.
Similar to high reward variance, miners will be discouraged to mine with high orphan blocks, which weakens the security of PoW-based consensus.

A possible approach that has already been deployed is rewarding orphan blocks instead of discarding them.
Ethereum~\cite{wood2014ethereum} adapts the GHOST protocol and introduces uncle blocks.
In Ethereum, orphan blocks can be included in the blockchain as uncle blocks, and miners finding uncle blocks will obtain mining reward, but less than normal blocks.

Another approach is to accelerate the block propagation to let miners know the latest block as soon as possible.
BIP-152~\cite{corallo2016bip} introduces the block relay protocol.
In the block relay protocol, miners synchronise their transaction pools in a real-time manner.
Once finding a block, the miner only broadcasts the compact version of it.
The compact block includes the block header and hashes of transactions in it, but does not include the transaction content.
Upon receiving a compact block, the miner reconstructs the block by matching transactions in his transaction pool with transaction hashes.
As the compact block is much smaller than the full block, broadcasting compact blocks can be much faster, and miners will be less possible to mine on earlier blocks.
Following BIP-152, several proposals on accelerating the block propagation have been proposed~\cite{ozisik2019graphene}\cite{klarman2018bloxroute}\cite{naumenko2019bandwidth}.


% \subsection{Why consensus requires mining}

% Nakamoto consensus requires PoW minig because it is designed that the consensus blockchain is the chain with the greatest PoW effort accumulated and therefore the greatest the greatest difficulty to produce, usually the longest chain.
% Such a chain is chosen as the majority agreement because it is the most difficult one to manipulate.

% Moreover, PoW can provide the blockchain with Sybil-resistance.
% Sybil attack is to subvert a reputation system by forging indentities.
% In PoW-style blockchains, the miner mines the new block is identically the leader producing the new block.
% That is to say, mining is to create the identity to be elected as the leader.
% As PoW requires computations, such a identity creation is neither costless.
% Therefore, PoW-style blockchains resist Sybil attacks by making identities generation expensive.

% \subsection{Why miners join mining}

% PoW-style blockchains incentivise miners to maintain the transaction history.

% Block subsidy and transaction fees are to encourage miners to keep recording new transaction.
% This help provide the blockchain with liveness: new blocks and valid transactions with appropriate feed will continue to be added to the ledger.

% The incentive also motivate the miners to stay honest and is to achieve incentive compatibility~\cite{}.
% Because the agreed chain has the corresponding PoW effort accumulated, to alter the history and double-spend his money, a miner needs to control a considerable amount of hash power to redo the PoW of the block, in which his first transaction was included, and all the following blocks since it.
% A miner owning such a computing power may choose between using it to double spend his money, or using it to mine new blocks and get rewarded. 

% To keep the blockchain system running expectedly, the reward needs to be profitable.
% Therefore, miners are incentivised to join mining.

% \subsection{Why miners join mining pools}

% As aforementioned, unsteady income stream will be risky, but the probability of a miner find a solution is in propotion to his hash power.
% Therefore, in pratice, miners often join a mining pool and perform computation on behalf of the pool, and get lower variant payouts allocated to them by the pool operators.

% This in fact relies on the fact that miners cannot steal the reward paying to the pool, because he neither can modify the coinbase transaction nor have knowledge on spending such an output;
% and the assumption that the pool will honestly pay to the miners according to their contribuitons.
% If a dishonest pool pays less to a miner than it deserves, the miner can discover it, leave the pool and shift to another.
% This will descrease the pool's mining power and lower his probability to mine a block, and no rational pool will do so.

% Therefore, the pool operator can trustlessly delegate the tasks to miners and miners tend to join pooled mining.
% However, this leads to the centralisation of computation power.

% \subsection{Balancing the sybil resistance and incentive}
% \label{sebsec:balancing}

% We have addressed in Section~\ref{sec:discourage-pool} that, our sheme completely discourage a rational pool operator who wants get mining reward.
% However, there still remains a concern that, we need to make a trade-off between Sybil resistance and incentivising miners to join mining.

% \HY{I find it difficult to connect it with Quadratic Voting or with Radical Market, so I skip it.}

% Assumes we have 2 different mining algorithms A and B, of which the computation-power-vs-investment-on-hardware curves are shown in Figure~\ref{fig:algo_A} and Figure~\ref{fig:algo_B} respectively.

% For algorithm A, with fewever investment on hardware, a miner can gain relativly higher computing power than B. That is, it is cheaper to generate identity, and hence it is less sybil-resistant.

% However, if using algorithm B, though it can bring stronger sybil resistance to the system, a miner will find that he cannot gain significant hash power unless he invest enough on the hardware.
% As aforementioned in Section~\ref{sec:intro}, low minig power will lead to unstable payouts, which will undoubtedly disincentivize miners if their budget are limited.

% Therefore, when designing the VRF, we need to take into account balancing the sybil resistance and incentive.

% \begin{figure}
% \centering
% \begin{tikzpicture}
%   \draw[->] (0,0) -- (6,0) node[right] {{Investment on Hardware}};
%   \draw[->] (0,0) -- (0,5) node[above] {{Hash Power}};
%   \draw (0,0) .. controls (3,0) and (4,0) .. (5,4.5);
% \end{tikzpicture}
% \caption{Algorithm A}
% \label{fig:algo_A}
% \end{figure}


% \begin{figure}
% \centering
% \begin{tikzpicture}
%   \draw[->] (0,0) -- (6,0) node[right] {{Investment on Hardware}};
%   \draw[->] (0,0) -- (0,5) node[above] {{Hash Power}};
%   \draw (0,0) .. controls (0,3) and (0,4) .. (5,4.5);
% \end{tikzpicture}
% \caption{Algorithm B}
% \label{fig:algo_B}
% \end{figure}
