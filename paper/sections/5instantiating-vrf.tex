\section{Instantiating VRF}


% 1. Choice of hash functions.
% 2. support of existing hash algos (hash should be slower than EC).

% Let $H_1(\cdot)$ be a hash function mapping an arbitrary-length string into an element in $G$. (str - elem)
%   rec: Elligator2
% Let $H_2(\cdot)$ be a hash function mapping an element in $G$ into a fixed-length string. (elem - str)
%   rec: toString then any hash
% Let $H_3(\cdot)$ be a hash function mapping an arbitrary-length string into a fixed-length string. (str - str)
%   fast hash for better verify performance

In order to implement VRF-based mining, one needs to first instantiate the VRF.
VRF has four configurable components, namely the elliptic curve and three hash functions $H_{1}(\cdot)$, $H_{2}(\cdot)$ and $H_{2}(\cdot)$.
$H_{1}(\cdot)$ maps an arbitrary-length string into an elliptic curve element;
$H_{2}(\cdot)$ maps an elliptic curve element into a fixed-length string; and
$H_{3}(\cdot)$ maps an arbitrary-length string into a fixed-length string.
In this section, we discuss considerations on choosing these four components for VRF-based mining.

\subsection{Elliptic curve}
As neither blockchains and VRF limits the choice of elliptic curves, any elliptic curve can be adapted.
Among prominent elliptic curves, Curve25519~\cite{} can be a promising choice.
Curve25519 supports Ed25519~\cite{}, a fast and secure digital signature algorithm.
In addition, numerous blockchains~\cite{} and projects using VRF~\cite{} adapt Ed25519 as their underlying elliptic curve.

\subsection{$H_{1}(\cdot)$}
$H_{1}(\cdot)$ is a hash-to-curve function.
% indistinguishable
Hash-to-curve functions should prevent distinguishing behaviours: adversaries cannot learn any pattern of the input from its hash.
% deterministic
In addition, the hash-to-curve function used in VRF should be deterministic, otherwise the hash will be unreproducible.
A standardisation document~\cite{} specifies several hash-to-curve functions that satisfy our requirements: Icart Method~\cite{}, Shallue-Woestijne-Ulas Method~\cite{}, Simplified SWU Method~\cite{} and Elligator2~\cite{}.
Elligator2 is the recommended hash-to-curve function for Curve25519.

\subsection{$H_{2}(\cdot)$}

\subsection{$H_{3}(\cdot)$}

\HY{being faster than hash is not a must-have property. but I agree that being too slow is undesired.}
\HY{becasue: different blockchains vary with hash algos;  if a hash algo is time-consuming per round, the network diff can still be adjusted to keep the block-time stable.}

\RH{this part we should discuss all 3 hash functions. Here we only talk about H2 (for producing hash). This might be tricky, so leave it the last one to do}

Existing mining algorithms can be easily integrated into VRF.

Currently, all existing mining algorithms take an input and then perform hash on it. Different algorithms vary with input length (depending on the blockchain data structure), and the hash function.
As shown in appendix~\ref{vrf_standardised_construction}, our construction accepts an arbitrary-length input $\alpha$.
We then simply use the hash function in the targeted mining algorithm as our $H_{1}$.
Thus, our scheme can add supports for different mining algorithms.
