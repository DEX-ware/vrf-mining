\section{Sybil-resistance v.s. incentive compatibility}

\subsection{Why consensus requires mining}

\RH{sybil attack}

% sybil vs eclipse

\subsection{Why miners join mining}

% bitcoin whitepaper

% incentive paper

\RH{mining can have rewards -> incentive}

\subsection{Why miners join mining pools}

\RH{1. pooled mining is possible, 2. pooled mining rewards more stable, ...}

% low computing power -> less likely to mine -> unstalbe in a period -> more centralised

\subsection{Balancing the sybil resistance and incentive}

\RH{There is a trade-off between sybil resistance and the incentive for miners to join the mining... Need some balance...}

% btc resists by pricing
% our construction: resistant but not stable
% less incentivised
% lower power

sybil attack is 
reputation system
forging identity

easier to find solution
easier to generate identity
less resistant

in bitcoin, leader, id, PoW, longest chain
The longest chain serves proof of transaction history caming from the largest pool of computing power.
As long as the majority of the computing power remain honest.
In bitcoin, the nodes are incentivised to maintain the history
the profit is low for misbehaving


easy to gen id -> full connection -> overwrite history


our construction
balance
cheaper to 51 and less sybil-resistant


We leave for future work quantitative study on a more formal model of the \TODO{.....}.