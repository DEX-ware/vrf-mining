\section{Sybil-resistance v.s. incentive compatibility}

\subsection{Why consensus requires mining}

Nakamoto consensus requires PoW minig because it is designed that the consensus blockchain is the chain with the greatest PoW effort accumulated and therefore the greatest the greatest difficulty to produce, usually the longest chain.
Such a chain is chosen as the majority agreement because it is the most difficult one to manipulate.

Moreover, PoW can provide the blockchain with Sybil-resistance.
Sybil attack is to subvert a reputation system by forging indentities.
In PoW-style blockchains, the miner mines the new block is identically the leader producing the new block.
That is to say, mining is to create the identity to be elected as the leader.
As PoW requires computations, such a identity creation is neither costless.
Therefore, PoW-style blockchains resist Sybil attacks by making identities generation expensive.

\subsection{Why miners join mining}

PoW-style blockchains incentivise miners to maintain the transaction history.

Block subsidy and transaction fees are to encourage miners to keep recording new transaction.
This help provide the blockchain with liveness: new blocks and valid transactions with appropriate feed will continue to be added to the ledger.

The incentive also motivate the miners to stay honest and is to achieve incentive compatibility~\cite{}.
Because the agreed chain has the corresponding PoW effort accumulated, to alter the history and double-spend his money, a miner needs to control a considerable amount of hash power to redo the PoW of the block, in which his first transaction was included, and all the following blocks since it.
A miner owning such a computing power may choose between using it to double spend his money, or using it to mine new blocks and get rewarded. 

To keep the blockchain system running expectedly, the reward needs to be profitable.
Therefore, miners are incentivised to join mining.

\subsection{Why miners join mining pools}

\RH{1. pooled mining is possible, 2. pooled mining rewards more stable, ...}

% low computing power -> less likely to mine -> unstalbe in a period -> more centralised

\subsection{Balancing the sybil resistance and incentive}

\RH{There is a trade-off between sybil resistance and the incentive for miners to join the mining... Need some balance...}

easier to find solution -> cheaper to 51 
easier to generate identity -> less sybil-resistant

% our construction: resistant but not stable

As aforementioned, unstable payouts will be risky.


Apparently, 
... is related to ...

Such a problem is known as Quadratic Voting~\cite{lalley2018quadratic}.
