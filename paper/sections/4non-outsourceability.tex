\section{Non-outsourceability analysis}
\label{sec:non_outsourceability}

% \cite{miller2015nonoutsourceable} first formally defines cryptocurrency mining as scratch-off puzzles. \TODO{describe}

% \begin{description}
%     \item[Correctness] Any valid hash can be verified.
%     \item[Feasibility] Mining is computationally feasible.
%     \item[Parallelisability] The mining process can be parallelised.
%     \item[$\mu$-Imcompressibility] The adversary cannot accelerate the mining process using pre-generated pairs of nonces and hashes, by at most a factor $\mu$ ($\mu \geq 1$). If $\mu = 1$, $\mathsf{Work}$ is the optimal way to mine.
%     \item[Non-transferability] Given a block template, the adversary cannot accelerate the mining process using other valid hashes on this block template.
% \end{description}

\subsection{Revised definitions}

Miller et al. \cite{miller2015nonoutsourceable} first formalises cryptocurrency mining as non-outsourceable scratch-off puzzles, and formally defines non-outsourceability.
In particular, they define two levels of non-outsourceability, namely \textbf{Weak Non-outsourceability} and \textbf{Strong Non-outsourceability}.

\begin{description}
    \item[Weak Non-outsourcability] If the pool operator outsources the mining process, miners can always steal the reward of mining.
    \item[Strong Non-outsourcability] In addition to the Weak Non-outsourcability, the pool operator cannot link the stolen mining reward with the miner who steals it.
\end{description}

Basically, \textbf{Weak Non-outsourceability} defines the punishment of outsourcing, while \textbf{Strong Non-outsourceability} covers both the punishment and the anonymity of malicious miners.
We call the property defining the punishment of outsourcing \textbf{Punish-mining-reward}.
The anonymity of malicious miners defined in~\cite{miller2015nonoutsourceable} is equivalent to \textbf{Transaction Unlinkability}~\cite{van2013cryptonote}: given two arbitrary transactions, distinguishing whether they have the same output address requires knowing the private keys holding their outputs.
To steal the mining reward, the miner should create a transaction, of which the input is the mining reward and the output is his owned address.
Identifying who steals the mining reward is equivalent to finding out who holds the output address, which can be further reduced to \textbf{Transaction Unlinkability}.


\subsection{Punish-secret-key-leakage}

% SK-non-outsourceability
We introduce \textbf{Punish-secret-key-leakage}, a new property of punishment on outsourcing that achieves stronger non-outsourceability than \textbf{Punish-mining-reward}.
A cryptocurrency mining protocol achieves \textbf{Punish-secret-key-leakage} in the following sense: the pool operator should reveal 1) the block template and 2) the private key receiving the coinbase transaction, so that miners can mine in the name of the pool operator.

% stronger than weak non-out
In a cryptocurrency mining protocol with \textbf{Punish-secret-key-leakage}, the pool operator should reveal his private key to miners, which gives miners opportunity to steal all cryptocurrency in the pool operator's wallet.
This indicates that, in terms of the punishment of outsourcing, \textbf{Punish-secret-key-leakage} is stronger than \textbf{Punish-mining-reward}, where miners can only steal the mining reward part.






\subsection{Non-outsourceability of VRF-based mining}

VRF-based mining achieves both \textbf{Punish-secret-key-leakage} and \textbf{Transaction Unlinkability}.
In VRF-based mining, the pool operator outsources mining requires revealing his private key to miners, which leads to \textbf{Punish-secret-key-leakage}.
Similar with the construction achieving \textbf{Strong Non-outsourceability} in~\cite{miller2015nonoutsourceable}, VRF-based mining achieves \textbf{Transaction Unlinkability} by allowing a malicious miner to use a freshly generated address to steal cryptocurrency.
To receive the stolen cryptocurrency of the pool operator, the malicious miner can create a new address, and construct a transaction of which the stolen cryptocurrency directs to this address.
As this new address has no historical transactions, linking the transaction stealing cryptocurrency with other transactions can be impossible.
Then the malicious miner can then spend his stolen cryptocurrency anonymously using numerous techniques, such as mixing services~\cite{maxwell2013coinjoin}\cite{bonneau2014mixcoin}\cite{ruffing2014coinshuffle}\cite{heilman2017tumblebit} and stealth addresses~\cite{van2013cryptonote}.


