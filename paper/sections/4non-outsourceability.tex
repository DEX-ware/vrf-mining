\section{Non-outsourceability}

% \cite{miller2015nonoutsourceable} first formally defines cryptocurrency mining as scratch-off puzzles. \TODO{describe}

% \begin{description}
%     \item[Correctness] Any valid hash can be verified.
%     \item[Feasibility] Mining is computationally feasible.
%     \item[Parallelisability] The mining process can be parallelised.
%     \item[$\mu$-Imcompressibility] The adversary cannot accelerate the mining process using pre-generated pairs of nonces and hashes, by at most a factor $\mu$ ($\mu \geq 1$). If $\mu = 1$, $\mathsf{Work}$ is the optimal way to mine.
%     \item[Non-transferability] Given a block template, the adversary cannot accelerate the mining process using other valid hashes on this block template.
% \end{description}

Miller et al. \cite{miller2015nonoutsourceable} first formalises cryptocurrency mining as non-outsourceable scratch-off puzzles, and formally defines non-outsourceability.
In particular, they define two levels of non-outsourceability, namely \textbf{Weak Non-outsourceability} and \textbf{Strong Non-outsourceability}.

\begin{description}
    \item[Weak Non-outsourcability] If the pool operator outsources the mining process, miners can always steal the reward of mining.
    \item[Strong Non-outsourcability] In addition to the Weak Non-outsourcability, the pool operator cannot link the stolen mining reward with the miner who steals it.
\end{description}

% SK-non-outsourceability
We introduce \textbf{SK-non-outsourceability}, a variant of non-outsourceability that is even more stronger than \textbf{Strong Non-outsourceability}. \HY{I am not sure if ``even more stronger'' is correct grammatically.}
A cryptocurrency mining protocol achieves \textbf{SK-non-outsourceability} in the following sense: the pool operator should reveal 1) the block template and 2) the private key associated with the public key in the coinbase transaction so that miners can mine in the name of the pool operator.

% stronger than weak non-out
In a cryptocurrency mining protocol with \textbf{SK-non-outsourceability}, the pool operator should reveal his private key to miners, which gives miners opportunity to steal all cryptocurrency in the pool operator's wallet.
This indicates that, in terms of the punishment of outsourcing, \textbf{SK-non-outsourceability} is stronger than \textbf{Weak Non-outsourceability}, where miners can only steal the mining reward part.

% anonymous
In addition, stealing behaviours in \textbf{SK-non-outsourceability} can be unaccountable.
To steal the cryptocurrency in the pool operator's wallet, a miner can construct a transaction, of which the input belongs to the pool operator's address and the output points to a new address generated by the miner.
As the new address has no transactions in history, tracing this address using blockchain data is impossible.
This achieves the same unaccountability level as \textbf{Strong Non-outsourceability}, where a miner can steal the mining reward without revealing his identity.




