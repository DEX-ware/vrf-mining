\section{Non-outsourceability analysis}
\label{sec:non_outsourceability}

In this section, we revisit existing definitions on \emph{non-outsourceability}, and show that VRF-based mining achieves strong \emph{non-outsourceability}.

\subsection{Revised definitions}

Miller et al. \cite{miller2015nonoutsourceable} first formalise cryptocurrency mining as \emph{scratch-off puzzles}, and formally define two levels of \emph{non-outsourceability}, namely \emph{weak non-outsourceability} and \emph{strong non-outsourceability}.

\begin{itemize}
    \item \emph{Weak non-outsourcability}: If the pool operator outsources the mining process, miners can always steal the reward of mining.
    \item \emph{Strong non-outsourcability}: In addition to \emph{weak non-outsourcability}, the pool operator cannot link the stolen mining reward with the miner who steals it.
\end{itemize}

\emph{Weak non-outsourceability} defines the punishment of outsourcing, while \emph{strong non-outsourceability} covers both the punishment and the anonymity of the stealer.
We call the property defining the punishment of outsourcing \emph{punish-mining-reward}.
The anonymity of stealers defined in~\cite{miller2015nonoutsourceable} is a special case of \emph{transaction unlinkability}~\cite{van2013cryptonote}: Given two arbitrary transactions, it is hard to know whether their outputs belong to the same secret key.
To steal the mining reward in \cite{miller2015nonoutsourceable}, the miner should create a transaction, of which the input is the mining reward and the output points to its address.
Finding out who steals the mining reward is equivalent to finding out the address holding the output, and we call the property making such attempt infeasible \emph{stealing-unlinkability}.




\subsection{Non-outsourceability of VRF-based mining}

VRF-based mining achieves \emph{punish-mining-reward}, but not \emph{stealing-unlinkability}.
To outsource mining to a miner, the pool operator should share a secret key with it.
This enables the miner to steal the mining reward, so VRF-based mining achieves \emph{punish-mining-reward}.
% linkability
The pool operator can give a new secret key and a block template to each miner at every height.
If a miner steals mining reward, the pool operator can identify the stealer by the secret key in the transaction stealing the reward.
Thus, VRF-based mining does not satisfy \emph{stealing-unlinkability}.

% expensive
Nevertheless, VRF-based mining makes maintaining mining pools quite expensive.
First, the pool operator should monitor stealing behaviours by examining new blocks' coinbase transactions all the time.
In addition, every time the pool operator's miners mine a block, the pool operator should transfer mining reward to its own wallet before a miner steals it.