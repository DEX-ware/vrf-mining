\section{Preliminaries}

\subsection{Cryptocurrency mining}

\RH{describe how PoW and mining works}



\subsection{Verifiable random functions}

Verifiable Random Function (VRF)~\cite{micali1999verifiable} is a public-key version of cryptographic hash function.
It involves a pair of a secret key and a public key.
Only the owner of the secret key can compute the hash, while anyone having the public key can verify the hash.
Formally, a VRF consists of four algorithms: $\mathsf{VRFKeyGen}$, $\mathsf{VRFHash}$, $\mathsf{VRFProve}$ and $\mathsf{VRFVerify}$.

\begin{itemize}
    \item $(sk, pk) \gets \mathsf{VRFKeyGen}(1^{\lambda})$: on input a security parameter $1^{\lambda}$, outputs the secret/public key pair $(sk, pk)$.
    \item $\beta \gets \mathsf{VRFHash}(sk, \alpha)$: on input $sk$ and an arbitrary-length string $\alpha$, outputs a fixed-length hash $\beta$.
    \item $\pi \gets \mathsf{VRFProve}(sk, \alpha)$: on input $sk$ and $\alpha$, outputs the proof $\pi$ for $\beta$.
    \item $\{0, 1\} \gets \mathsf{VRFVerify}(pk, \alpha, \beta, \pi)$: on input $pk$, $\alpha$, $\beta$, $\pi$, outputs the verification result 0 or 1.
\end{itemize}