\documentclass[runningheads]{llncs}

\usepackage{graphicx}
\usepackage{xcolor}
\usepackage{interval}
\usepackage{amsmath, amssymb}
\usepackage{cuted, tcolorbox}
\usepackage{tikz}
\usepackage[ruled,vlined]{algorithm2e}
\usepackage{algpseudocode}

\usepackage[normalem]{ulem}

\makeatletter
\newcommand{\printfnsymbol}[1]{%
  \textsuperscript{\@fnsymbol{#1}}%
}
\makeatother

\newcommand{\RH}[1]{\textcolor{blue}{#1}}
\newcommand{\JS}[1]{\textcolor{red}{#1}}
\newcommand{\TODO}[1]{\textcolor{red}{TODO: #1}}
\newcommand{\HY}[1]{\textcolor{brown}{#1}}

% blue comment
\newcommand\mycommfont[1]{\footnotesize\ttfamily\textcolor{blue}{#1}}
\SetCommentSty{mycommfont}
% blue triangle
\SetKwComment{Comment}{\color{blue} $\triangleright$\ }{}

\begin{document}

\title{VRF-based mining: How to rule out mining pools from blockchain ecosystem?}
\titlerunning{VRF-based mining}

\author{
    Runchao Han\thanks{equal contribution}\inst{1,2}
    \and Haoyu Lin\printfnsymbol{1}
    \and Jiangshan Yu\inst{1}
}
\authorrunning{Han et al.}

\institute{Monash University \and CSIRO-Data61}

\maketitle

\begin{abstract}
    TODO
    \keywords{Blockchain \and Cryptocurrency mining \and Mining pool.}
\end{abstract}

\section{Introduction}
\label{sec:intro}

Bitcoin~\cite{nakamoto2008bitcoin} started the era of cryptocurrency.
Bitcoin's main novelty is Nakamoto consensus - the first consensus protocol that can work in permissionless settings.
% blockchain
In Nakamoto consensus, each participant maintains its own ledger, and the ledger is usually called blockchain.
A blockchain is formed as a chain of blocks, and each block contains a batch of transactions.
Blockchain is designed to be append-only, and its integrity is protected by digital signatures.
% proof of work
Participants keep appending blocks to the blockchain.
To append a block, a participant should first solve a Proof-of-Work (PoW)~\cite{dwork1992pricing} puzzle - a probabilistic and moderately hard computation puzzle.
Once solving the puzzle, the participant will append his block to the blockchain, and get some cryptocurrency as reward.
By adaptively adjusting the difficulty of PoW puzzles, Nakamoto consensus can stabilise the speed of generating blocks.
Participants of Nakamoto consensus are also known as miners, the process of solving PoW puzzles is known as mining, and the computation power used for mining is known as mining power.

% why pooled mining
Today, mining pools dominate the mining power of most blockchains using Nakamoto consensus.
Mining pool is a kind of service that gathers mining power from solo miners and rewards solo miners in a more fine-grained way.
More specifically, a single blockchain may consist of numerous miners, and each miner may have a low chance to mine the next block, leading to unstable mining reward.
A mining pool allows miners to mine on blockchain in the name of the pool operator, and the pool operator distributes the mining reward to solo miners according to their contributed mining power.
In this way, a miner can get reward more stably by joining a mining pool.

% Problems of having mining pools
However, mining pools lead to centralisation of mining power, which contradicts the design objective of Bitcoin - the decentralisation.
% problem of centralisation
Centralisation of mining power can be harmful for the security of Nakamoto consensus.
A mining pool with sufficient mining power can perform numerous types of attacks to break the consensus, such as selfish mining attacks~\cite{eyal2018majority} and 51\% attacks~\cite{nakamoto2008bitcoin}.
Even top mining pools are not big enough, multiple mining pools can even collude to launch attacks.
In addition, mining pools can decide which transactions to include in order to censor the blockchain.
% to date...
To date (01/12/2019), four largest mining pools control more than 51\% Bitcoin mining power, which can be a lurking threat of Bitcoin.




\textbf{Our contributions.}
In this work, we put forward a novel idea of combining VRF and mining in PoW to eliminate the intentionality of pooled mining, and to conquer the centralisation of mining power. We call our construction \textbf{VRF-based mining}.

\TODO{We should make the contribution part more sophisticated here. At least we have the following contributions:
Main: We propose the VRF-based mining that makes pooled mining impossible. In particular, we
1. Propose the detailed construction of VRF-based mining, which can be a drop-in replacement of existing hash-based mining (we already have)
2. We rethink the desired properties of mining that can make pooled mining difficult while remaining secure (Runchao will do this later)
3. We justify the security and non-outsourcability of VRF-based mining against existing attacks and vulnerabilities (Runchao)
4. We analyse cons of having no mining pool and possible solutions (Haoyu)
5. We discuss how to discourage pooled staking using VRF (Haoyu)
}

\textbf{Paper organisation.}
% \TODO{we should have a paragraph describing the organisation of this paper. See the paper ``New Techniques for Efficient Trapdoor Functions and Applications''}
We revisit the concepts of mining, the standard definitions and the properties of VRF and EC-VRF in Section~\ref{sec:preliminaries}.
We give our construction of VRF-based mining in Section~\ref{sec:construction} and \ref{sec:instantiation},
and then analyse its non-outsourceability in Section~\ref{sec:non_outsourceability}.
Finally, we discuss the \TODO{XXX} in Section~\ref{sec:problems-and-solutions},
and the related work in Section~\ref{sec:related}.

\section{Preliminaries}

\subsection{Cryptocurrency mining}

In cryptocurrency mining, miners search for the solution to a computational puzzle.
Specifically, the puzzle is to find a (partial) pre-image for a cryptographic hash function, of which output is less than a target value begining with several consecutive zero bits.
The hash function maps a list of transaction, the hash of the previous block, a timestamp, an nonce and other information, to a result value.
As the hash result is unpredictable and unformly distributed, the probability of the hash result lies among the desired interval is inversely proportional to the number of leading zero.

The more computation power the miner possesses, the more rounds of hash he can run during a time unit, and the more likely he can find the solution. The first miner who finds the solution and anounces it, wins the competition and gets rewarded.

\subsection{Verifiable random functions}

Verifiable Random Function (VRF)~\cite{micali1999verifiable} is a public-key version of cryptographic hash function.
It involves a pair of a secret key and a public key.
Only the owner of the secret key can compute the hash, while anyone having the public key can verify the hash.
Formally, a VRF consists of four algorithms: $\mathsf{VRFKeyGen}$, $\mathsf{VRFHash}$, $\mathsf{VRFProve}$ and $\mathsf{VRFVerify}$.

\begin{itemize}
    \item $(sk, pk) \gets \mathsf{VRFKeyGen}(1^{\lambda})$: on input a security parameter $1^{\lambda}$, outputs the secret/public key pair $(sk, pk)$.
    \item $\beta \gets \mathsf{VRFHash}(sk, \alpha)$: on input $sk$ and an arbitrary-length string $\alpha$, outputs a fixed-length hash $\beta$.
    \item $\pi \gets \mathsf{VRFProve}(sk, \alpha)$: on input $sk$ and $\alpha$, outputs the proof $\pi$ for $\beta$.
    \item $\{0, 1\} \gets \mathsf{VRFVerify}(pk, \alpha, \beta, \pi)$: on input $pk$, $\alpha$, $\beta$, $\pi$, outputs the verification result 0 or 1.
\end{itemize}


\section{VRF-based mining}

\subsection{Our construction}

We replace hash function in mining by VRF, and thus construct VRF-based mining.

To rule out pooled mining, we combine VRF with digital signature scheme.
As the digital signature scheme use in cryptocurrency are based on Elliptic curve, we construct Elliptic curve based VRF for our VRF-based mining.
The standardised construction is put in the appendix~\ref{vrf_standardised_construction}.

% The desired properties of a well-designed hash function includes uniformity, deterministic, unpredictability and verifiability.
% As shown in appendix~\ref{vrf_standardised_construction}, ...
% Therefore, VRF fits in here.

In the case of solo mining, our sheme works as follows.

\TODO{change format to algorithm2e}
% https://www.overleaf.com/learn/latex/algorithms
\begin{enumerate}
    \item The solo miner runs $\mathsf{VRFKeyGen}(1^{\lambda})$ to generate a secret key and public key pair $(sk, pk)$.
    \item The miner queries the the full node it runs, and build a block template, including a coinbase transaction paying to him, for example, to his \texttt{scriptPubKey}.
    \item The miner alters the nonce in the block header.
    \item The miner takes the block header as $\alpha$, and runs $\mathsf{VRFHash}(sk, \alpha)$ to get $\beta$. If $\beta$ meets the difficulty requirement, a block is found; otherwise, the miner changes the nonce and runs $\mathsf{VRFHash}(sk, \alpha)$ again.
    \item The miner runs $\mathsf{VRFProve}(sk, \alpha)$ to generate a proof $\pi$.
    \item The miner then submit the $pk$, $\beta$, and the proof $\pi$ to the node.
    \item The node then check:
        \begin{enumerate}
            \item if $\beta$ meets the difficulty requirement;
            \item if $\mathsf{VRFVerify}(pk, \alpha, \beta, \pi)$ result is valid;
            \item whether the \texttt{scriptPubKey} in the coinbase transaction corresponds to $pk$.
        \end{enumerate}
        If all results are valid, a new block is mined, and the coinbase reward is paid to the miner.
    \item $pk$, $\beta$ and $\pi$ should also be recorded in the block header, so that, when the block is propogated to other nodes, they can easily validate it via $\mathsf{VRFVerify}(pk, \alpha, \beta, \pi)$.
\end{enumerate}

\subsection{Why pooled mining cannot work in our scheme}

% need private key
As shown in appendix~\ref{vrf_standardised_construction}, both $\mathsf{VRFHash}$ and $\mathsf{VRFProve}$ take secret key $sk$ as input.
That is to say, if a pool operator would like to provide pooled mining service and let miners participate, he will need to provide miners with his secret key.

% none would like to provide private key
However, none rational pool operator will tend to give out his secret key, as revealing his secret key give others the opportunity to redeem his balance or forge his identity.

Therefore, following our construction, we eliminate the intentionality of pooled mining and can reduce the centralisation of mining power significanlty.

\section{Non-outsourceability analysis}
\label{sec:non_outsourceability}

% \cite{miller2015nonoutsourceable} first formally defines cryptocurrency mining as scratch-off puzzles. \TODO{describe}

% \begin{description}
%     \item[Correctness] Any valid hash can be verified.
%     \item[Feasibility] Mining is computationally feasible.
%     \item[Parallelisability] The mining process can be parallelised.
%     \item[$\mu$-Imcompressibility] The adversary cannot accelerate the mining process using pre-generated pairs of nonces and hashes, by at most a factor $\mu$ ($\mu \geq 1$). If $\mu = 1$, $\mathsf{Work}$ is the optimal way to mine.
%     \item[Non-transferability] Given a block template, the adversary cannot accelerate the mining process using other valid hashes on this block template.
% \end{description}

\subsection{Revised definitions}

Miller et al. \cite{miller2015nonoutsourceable} first formalises cryptocurrency mining as non-outsourceable scratch-off puzzles, and formally defines non-outsourceability.
In particular, they define two levels of non-outsourceability, namely \textbf{Weak Non-outsourceability} and \textbf{Strong Non-outsourceability}.

\begin{description}
    \item[Weak Non-outsourcability] If the pool operator outsources the mining process, miners can always steal the reward of mining.
    \item[Strong Non-outsourcability] In addition to the Weak Non-outsourcability, the pool operator cannot link the stolen mining reward with the miner who steals it.
\end{description}

Basically, \textbf{Weak Non-outsourceability} defines the punishment of outsourcing, while \textbf{Strong Non-outsourceability} covers both the punishment and the anonymity of malicious miners.
We call the property defining the punishment of outsourcing \textbf{Punish-mining-reward}.
The anonymity of malicious miners defined in~\cite{miller2015nonoutsourceable} is equivalent to \textbf{Transaction Unlinkability}~\cite{van2013cryptonote}: given two arbitrary transactions, distinguishing whether they have the same output address requires knowing the private keys holding their outputs.
To steal the mining reward, the miner should create a transaction, of which the input is the mining reward and the output is his owned address.
Identifying who steals the mining reward is equivalent to finding out who holds the output address, which can be further reduced to \textbf{Transaction Unlinkability}.


\subsection{Punish-secret-key-leakage}

% SK-non-outsourceability
We introduce \textbf{Punish-secret-key-leakage}, a new property of punishment on outsourcing that achieves stronger non-outsourceability than \textbf{Punish-mining-reward}.
A cryptocurrency mining protocol achieves \textbf{Punish-secret-key-leakage} in the following sense: the pool operator should reveal 1) the block template and 2) the private key receiving the coinbase transaction, so that miners can mine in the name of the pool operator.

% stronger than weak non-out
In a cryptocurrency mining protocol with \textbf{Punish-secret-key-leakage}, the pool operator should reveal his private key to miners, which gives miners opportunity to steal all cryptocurrency in the pool operator's wallet.
This indicates that, in terms of the punishment of outsourcing, \textbf{Punish-secret-key-leakage} is stronger than \textbf{Punish-mining-reward}, where miners can only steal the mining reward part.






\subsection{Non-outsourceability of VRF-based mining}

VRF-based mining achieves both \textbf{Punish-secret-key-leakage} and \textbf{Transaction Unlinkability}.
In VRF-based mining, the pool operator outsources mining requires revealing his private key to miners, which leads to \textbf{Punish-secret-key-leakage}.
Similar with the construction achieving \textbf{Strong Non-outsourceability} in~\cite{miller2015nonoutsourceable}, VRF-based mining achieves \textbf{Transaction Unlinkability} by allowing a malicious miner to use a freshly generated address to steal cryptocurrency.
To receive the stolen cryptocurrency of the pool operator, the malicious miner can create a new address, and construct a transaction of which the stolen cryptocurrency directs to this address.
As this new address has no historical transactions, linking the transaction stealing cryptocurrency with other transactions can be impossible.
Then the malicious miner can then spend his stolen cryptocurrency anonymously using numerous techniques, such as mixing services~\cite{maxwell2013coinjoin}\cite{bonneau2014mixcoin}\cite{ruffing2014coinshuffle}\cite{heilman2017tumblebit} and stealth addresses~\cite{van2013cryptonote}.



\section{Instantiating VRF}

In order to implement VRF-based mining, one needs to first instantiate the VRF.
VRF has four configurable components, namely the elliptic curve and three hash functions $H_{1}(\cdot)$, $H_{2}(\cdot)$ and $H_{2}(\cdot)$.
$H_{1}(\cdot)$ maps an arbitrary-length string into an elliptic curve element;
$H_{2}(\cdot)$ maps an elliptic curve element into a fixed-length string; and
$H_{3}(\cdot)$ maps an arbitrary-length string into a fixed-length string.
In this section, we discuss considerations on choosing these four components for VRF-based mining.





\subsection{Elliptic curve}

As neither blockchains and VRF limits the choice of elliptic curves, any elliptic curve can be adapted.
Among prominent elliptic curves, Curve25519~\cite{} can be a promising choice.
Curve25519 supports Ed25519~\cite{}, a fast and secure digital signature algorithm.
In addition, numerous blockchains~\cite{} and projects using VRF~\cite{} adapt Ed25519 as their underlying elliptic curve.




\subsection{$H_{1}(\cdot)$ (hashing a string to an elliptic curve point)}

% indistinguishable
$H_{1}(\cdot)$ is a hash-to-curve function, which should prevent distinguishing behaviours: adversaries cannot learn any pattern of the input from its hash.
% deterministic
In addition, the hash-to-curve function used in VRF should be deterministic, otherwise the hash will be unreproducible.

A standardisation document~\cite{} specifies several hash-to-curve functions that fit into different elliptic curves and satisfy our requirements: Icart Method~\cite{}, Shallue-Woestijne-Ulas Method~\cite{}, Simplified SWU Method~\cite{} and Elligator2~\cite{}.
Within these hash-to-curve functions, Elligator2 is the recommended one for Curve25519.




\subsection{$H_{2}(\cdot)$ (hashing an elliptic curve point to a string)}

$H_{2}(\cdot)$ hashes an elliptic curve point to a fixed-length string.
It can be divided to two steps: 1) encoding an elliptic curve point to a string, and 2) hashing the string using a normal hash function.
The encoding step can be bidirectional and unencrypted, so can be done simply by converting a big integer to a string.
The hashing step should be cryptographically secure, so can adapt any existing cryptographically secure hash functions.

For ASIC-resistant cryptocurrencies using memory-hard hash functions (e.g., Ethereum~\cite{} and Monero~\cite{}), there are some concerns on choosing hash functions in $H_{2}(\cdot)$.
% VRFHash shoule memory-hard
To make mining memory-hard, $\mathsf{VRFHash}$ should be memory-hard.
% vrfhash steps
$\mathsf{VRFHash}$ of the standardised VRF consists of one $H_{1}(\cdot)$ hashing, one scalar-point multiplication and one $H_{2}(\cdot)$ hashing.
$H_{1}(\cdot)$ and the encoding step in $H_{2}(\cdot)$ can be fast so less possible to become the performance hotspot.
The scalar-point multiplication is slow and computation (rather than memory)-intensive.
% make VRFHash memory hard
The hash function in $H_{2}(\cdot)$ is the only possibility to make $\mathsf{VRFHash}$ memory-hard, and it should be overwhelmingly slower than the scalar-point multiplication, otherwise the scalar-point multiplication will make $\mathsf{VRFHash}$ computation-intensive.
This requirement can be achieved by repetitively executing one or multiple different memory-hard hash functions (such as Ethash~\cite{}, Equihash~\cite{}, and CryptoNight~\cite{}).





\subsection{$H_{3}(\cdot)$ (Normal hash function)}

$H_{3}(\cdot)$ is only used in $\mathsf{VRFProve}$ (proving the authorship of hashes) and $\mathsf{VRFVerify}$ (verifying the authorship of hashes).
The overhead of proving and verification should be minimised, so cryptographically secure hash functions that are designed to be fast (such as Keccak~\cite{} and BLAKE~\cite{}) are suitable for $H_{3}(\cdot)$.
\section{Problems and possible solutions}
\label{sec:problems-and-solutions}

\RH{I have thought about this section, and found my previous conclusion and story are wrong.
Instead, we should discuss the disadvantage of having no mining pools:
1. without pools, miners are rewarded unstably, so are discouraged to mine, so mining power of the blockchain will be less, so weaker security guarantee.
2. high orphan block rate, network may have greater impact on mining.
Solutions for 1: silumate mining pool on blockchain. This should be discussed online, and there has been a paper describing collaborative mining (https://www.usenix.org/system/files/sec19-szalachowski.pdf)
Solutions for 2: longer blocktime, Grephene/propogate headers only, and other layer0 solutions...}








\subsection{Why consensus requires mining}

Nakamoto consensus requires PoW minig because it is designed that the consensus blockchain is the chain with the greatest PoW effort accumulated and therefore the greatest the greatest difficulty to produce, usually the longest chain.
Such a chain is chosen as the majority agreement because it is the most difficult one to manipulate.

Moreover, PoW can provide the blockchain with Sybil-resistance.
Sybil attack is to subvert a reputation system by forging indentities.
In PoW-style blockchains, the miner mines the new block is identically the leader producing the new block.
That is to say, mining is to create the identity to be elected as the leader.
As PoW requires computations, such a identity creation is neither costless.
Therefore, PoW-style blockchains resist Sybil attacks by making identities generation expensive.

\subsection{Why miners join mining}

PoW-style blockchains incentivise miners to maintain the transaction history.

Block subsidy and transaction fees are to encourage miners to keep recording new transaction.
This help provide the blockchain with liveness: new blocks and valid transactions with appropriate feed will continue to be added to the ledger.

The incentive also motivate the miners to stay honest and is to achieve incentive compatibility~\cite{}.
Because the agreed chain has the corresponding PoW effort accumulated, to alter the history and double-spend his money, a miner needs to control a considerable amount of hash power to redo the PoW of the block, in which his first transaction was included, and all the following blocks since it.
A miner owning such a computing power may choose between using it to double spend his money, or using it to mine new blocks and get rewarded. 

To keep the blockchain system running expectedly, the reward needs to be profitable.
Therefore, miners are incentivised to join mining.

\subsection{Why miners join mining pools}

As aforementioned, unsteady income stream will be risky, but the probability of a miner find a solution is in propotion to his hash power.
Therefore, in pratice, miners often join a mining pool and perform computation on behalf of the pool, and get lower variant payouts allocated to them by the pool operators.

This in fact relies on the fact that miners cannot steal the reward paying to the pool, because he neither can modify the coinbase transaction nor have knowledge on spending such an output;
and the assumption that the pool will honestly pay to the miners according to their contribuitons.
If a dishonest pool pays less to a miner than it deserves, the miner can discover it, leave the pool and shift to another.
This will descrease the pool's mining power and lower his probability to mine a block, and no rational pool will do so.

Therefore, the pool operator can trustlessly delegate the tasks to miners and miners tend to join pooled mining.
However, this leads to the centralisation of computation power.

\subsection{Balancing the sybil resistance and incentive}
\label{sebsec:balancing}

We have addressed in Section~\ref{sec:discourage-pool} that, our sheme completely discourage a rational pool operator who wants get mining reward.
However, there still remains a concern that, we need to make a trade-off between Sybil resistance and incentivising miners to join mining.

\HY{I find it difficult to connect it with Quadratic Voting or with Radical Market, so I skip it.}

Assumes we have 2 different mining algorithms A and B, of which the computation-power-vs-investment-on-hardware curves are shown in Figure~\ref{fig:algo_A} and Figure~\ref{fig:algo_B} respectively.

For algorithm A, with fewever investment on hardware, a miner can gain relativly higher computing power than B. That is, it is cheaper to generate identity, and hence it is less sybil-resistant.

However, if using algorithm B, though it can bring stronger sybil resistance to the system, a miner will find that he cannot gain significant hash power unless he invest enough on the hardware.
As aforementioned in Section~\ref{sec:intro}, low minig power will lead to unstable payouts, which will undoubtedly disincentivize miners if their budget are limited.

Therefore, when designing the VRF, we need to take into account balancing the sybil resistance and incentive.

\begin{figure}
\centering
\begin{tikzpicture}
  \draw[->] (0,0) -- (6,0) node[right] {{Investment on Hardware}};
  \draw[->] (0,0) -- (0,5) node[above] {{Hash Power}};
  \draw (0,0) .. controls (3,0) and (4,0) .. (5,4.5);
\end{tikzpicture}
\caption{Algorithm A}
\label{fig:algo_A}
\end{figure}


\begin{figure}
\centering
\begin{tikzpicture}
  \draw[->] (0,0) -- (6,0) node[right] {{Investment on Hardware}};
  \draw[->] (0,0) -- (0,5) node[above] {{Hash Power}};
  \draw (0,0) .. controls (0,3) and (0,4) .. (5,4.5);
\end{tikzpicture}
\caption{Algorithm B}
\label{fig:algo_B}
\end{figure}

\section{Ruling out staking pools}
\section{Related work}

To the best of our knowledge, VRF-based mining is the first construction that makes pooled mining \textit{impossible}.
In this section, we briefly review related research on preventing mining pools, and compare them with VRF-based mining.
We classify related research to two types, namely mining protocols trying to address pooled mining, and decentralised mining pools.

\subsection{Mining protocols}

There are two mining protocols aiming at discouraging or breaking mining pools: the \textit{non-outsourceable scratch-off puzzle}~\cite{miller2015nonoutsourceable} and the \textit{Two Phase Proof-of-Work} (\textit{2P-PoW})~\cite{2P-PoW}.

\textbf{Non-outsourceable scratch-off puzzle.}
Miller et al.~\cite{miller2015nonoutsourceable} formalises cryptocurrency mining as \textit{Scratch-off puzzles}, defines \textit{Non-outsourceable scratch-off puzzles}, and proposes two constructions.
As discussed in \S\ref{sec:non_outsourceability}, compared to the non-outsourceable scratch-off puzzle, our VRF-based mining achieves better non-outsourceability, and is much simpler to implement.

\textbf{2P-PoW.}
Eyal and Sirer proposes 2P-PoW~\cite{2P-PoW}, a mining protocol that discourages pooled mining.
In 2P-PoW, there are two phases, and each phase has a difficulty parameter.
A miner should find a nonce that makes the block to pass two phases: 1) the Sha256d hash of the block meets the first difficulty, 2) the Sha256 hash of the signature of the block meets the second difficulty.
As the second requirement requires the private key, pool operators cannot outsource the second phase.

Compared to VRF-based mining that makes pooled mining impossible, 2P-PoW only discourages pooled mining, as the first phase is outsourceable.
In addition, 2P-PoW should use deterministic digital signatures, while commonly used digital signatures (e.g., ECDSA, EdDSA) rely on randomisation.
If the signature is non-deterministic, the pool operator can make use of all nonces that are generated by miners and meet the first difficulty.
For example, given a nonce meeting the first difficulty, the pool operator repetitively generates signatures to meet the second requirement.
Moreover, how to adjust two difficulties still remains unknown and requires some simulations.





\subsection{Decentralised mining pools}

% ref: https://www.alexeizamyatin.me/files/Decentralized_Mining-Security_and_Attacks.pdf


Decentralised mining pool is a type of mining pool that works in a decentralised way.
More specifically, miners mine in the name of themselves rather than the pool operator, but they share reward in a fine-grained way.
In this way, miners are rewarded stably while mining power is not aggregated to pool operators.



\textbf{P2Pool}~\cite{p2pool} is a decentralised mining pool for Bitcoin.
% how p2pool works
P2Pool runs a share-chain among all miners in the pool, and the share-chain includes shares submitted to the pool in sequence.
During mining, the coinbase transaction records the number of shares submitted by each miner, and distributes the mining reward according to miners' contribution.
In this way, once mining a Bitcoin block, the coinbase transaction will become valid, and miners will be rewarded according to their contribution.
% challenges
However, P2Pool suffers from several challenges.
First, handling the difficulty of mining shares in P2Pool is hard.
If the difficulty is high, a miner's reward will still be volatile.
If the difficulty is low, there will be numerous low-difficulty shares, which introduces huge overhead on broadcasting shares or even spamming attacks.
Second, frequent share submissions amplifies the influence of network latency which leads to high orphan share rate.
Last, P2Pool is vulnerable to temporary dishonest majority~\cite{decentralised-mining-pool-security}.

\textbf{SmartPool}~\cite{luu2017smartpool} is another decentralised mining pool, which uses a smart contract to replace the centralised pool operator.
% con 1: smart contract
As relying on smart contracts, SmartPool cannot work on blockchains without smart contracts.
% con 2: network delay
In addition, as blockchains achieve limited throughput, transaction processing might be congested and mining reward might be delayed, especially when a large number of miners participate in the SmartPool.
% con 3: compute and storage overhead
Moreover, as the SmartPool smart contract should verify the validity of blocks, miners should submit the whole block (with transactions) to the SmartPool.
Verifying blocks introduces significant overhead on computing (so expensive transaction fees), and storing blocks in the SmartPool smart contract also introduces significant overhead on storing the blockchain.


\section{Conclusion}

\TODO{revise after finishing everything}

\TODO{add a Discussion section before this section}

In this paper, we remind people the centralised hash power distribution in Bitcoin and the subsequences.
We then construct VRF-based mining to resolve the centralised pooled mining criticism,
with a following discussion on balancing sybil-resistance and incentivisation.
We also analyse the candidate solutions to mining power centralisation, and point out that unfortunately none of them prevents pooled mining.

\bibliographystyle{splncs04}
\bibliography{refs}

\appendix
\section{Standardised construction of VRF}

Let $G$ be a cyclic group of prime order $q$ with generator $g$. Assume $E = G$
Let $H_1(\cdot)$ be a hash function mapping an arbitrary-length string into an element in $G$.
Let $H_2(\cdot)$ be a hash function mapping an element in $G$ into a fixed-length string.
Let $H_3(\cdot)$ be a hash function mapping an arbitrary-length string into a fixed-length string.

$(sk, pk) \leftarrow VRFKeyGen(1^{\lambda})$
\begin{enumerate}
    \item $sk$ is chosen uniformly and randomly from $\interval{0}{q-1}$
    \item $pk = g^{sk}$
\end{enumerate}

$h \leftarrow VRFHash(sk, \alpha)$
\begin{enumerate}
    \item $h = H_{1}(\alpha)$
    \item $\gamma = h^{sk}$
    \item $\beta = H_{2}(\gamma)$
\end{enumerate}

$\pi \leftarrow VRFProve(sk, \alpha)$
\begin{enumerate}
    \item $h = H_{1}(\alpha)$
    \item $\gamma = h^{sk}$
    \item choose a random $k \in \interval{0}{q-1}$
    \item $c = H_{3}(g, h, pk, h^{sk}, g^{k}, h^{k})$
    \item $s = k - c \cdot sk \pmod{q}$
    \item $\pi = (\gamma, c, s)$
\end{enumerate}

$\{0, 1\} \leftarrow VRFVerify(pk, \alpha, \beta, \pi)$
\begin{enumerate}
    \item $u = pk^{c} \cdot g^{s}$
    \item $h = H_{1}(\alpha)$
    \item check if $\gamma \in G$
    \item $v = \gamma^{c} \cdot h^{s}$
    \item check if $c \stackrel{?}{=} H_{3}(g, h, pk, \gamma, u, v)$
\end{enumerate}


\end{document}