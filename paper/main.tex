\documentclass[runningheads]{llncs}

\usepackage{graphicx}
\usepackage{xcolor}
\usepackage{interval}
\usepackage{amsmath, amssymb}
\usepackage{cuted, tcolorbox}
\usepackage{tikz}
\usepackage[ruled,vlined]{algorithm2e}
\usepackage{algpseudocode}

\usepackage[normalem]{ulem}

\makeatletter
\newcommand{\printfnsymbol}[1]{%
  \textsuperscript{\@fnsymbol{#1}}%
}
\makeatother

\newcommand{\RH}[1]{\textcolor{blue}{#1}}
\newcommand{\JS}[1]{\textcolor{red}{#1}}
\newcommand{\TODO}[1]{\textcolor{red}{TODO: #1}}
\newcommand{\HY}[1]{\textcolor{brown}{#1}}

% blue comment
\newcommand\mycommfont[1]{\footnotesize\ttfamily\textcolor{blue}{#1}}
\SetCommentSty{mycommfont}
% blue triangle
\SetKwComment{Comment}{\color{blue} $\triangleright$\ }{}

\begin{document}

\title{VRF-based mining: How to rule out mining pools from blockchain ecosystem?}
\titlerunning{VRF-based mining}

\author{
    Runchao Han\thanks{equal contribution}\inst{1,2}
    \and Haoyu Lin\printfnsymbol{1}
    \and Jiangshan Yu\inst{1}
}
\authorrunning{Han et al.}

\institute{Monash University \and CSIRO-Data61}

\maketitle

\begin{abstract}
    TODO
    \keywords{Blockchain \and Cryptocurrency mining \and Mining pool.}
\end{abstract}

\section{Introduction}
\label{sec:intro}

% Bitcoin
By December 2019, Bitcoin is the most popular cryptocurrency in the world.
It uses a public ledger to record the transaction history: transactions are stored in blocks, and each block is linked to its previous block.
In Bitcoin, Proof of Work (PoW) is used to timestamp transactions, and to determine the majority agreement on the transaction history.
% mining
Miners search for a PoW solution to a puzzle concurrently, and the miner who outruns others becomes the leader to produce the next block and win the reward.
Such a reward is to incentivise the miners to maintain the ledger and maintain it honestly.
Though in Proof of Stake~\cite{} and its variants, it is similar that coins are minted and rewarded to the miner for each block, it requires no hash calculation to perform puzzle solution searching.
Therefore, we only discuss mining in PoW in this paper.

Because of the hash results are distributed uniformly and cannot be predicted, the probabilty of a miner to find a solution depends on the hash power it ownes.
Therefore, payouts will be less frequent for a solo miner controlling low computing power.
As miners need to pay for electricity periodically and want to estimate the remaining time to get the investment on hardware back (if they have not), inconsistent payouts can be risky for miners.

% mining pool
To reduce the risk, miners join mining pool to complete the computation tasks.
This also saves miners' effort on running full node and dealing with block creation, block propogation and other technical details.
Miners can then focus on running mining rigs, for example, ASICs.

Pool operator is then responsible to run full node, distribute the computing task to miners, 
produce a block if the submitted solution found meeting work requirement.
If the solution does not meet the block work requirement but a reduced requirement, which is called ``pool difficulty'', the miner can also get paid, depending on its mining power, because it helps eliminate incorrect answers.
Thus, mining pool can gather considerable amount of computation power, gain higher probability to mine a block and pay to miners consistently.

% centralisation
However, mining in Bitcoin network suffers from centralisation.
By December 2019, the 4 largest mining pool, put together, control more than 51\% of the hash power in Bitcoin network~\cite{}, whereas there are more than 29 mining pools known~\cite{}.

The drawbacks of mining power centralisation is non-negligible.
An unexhausted list is shown as follows.
\begin{itemize}
    % 51%
    \item Pool operators can collude with each other, perform 51\% attacks~\cite{} and double spend their money, if the sum of their hash power is greater than 51\% of the entire network. \HY{do we need to talk about selfish mining?}
    % failure
    \item Centralisation introduces points of failure into the network: if a big mining pool is hacked, or DNS is spoofed, a considerable amount of mining power can be stolen.
    % censorship
    \item Mining pools can perform censorship on the transaction, rejecting transactions from some exchanges, or CoinJoin transactions.
    % direct
    \item A pool can also direct the hash power it controls to mine which fork he decides to extend, and even the worse, to mine another coin if using the same hash algorithm, which is undoubtedly harmful to the ecosystem.
    % upgrade
    \item Pool operators may stall protocol upgrade, for example, Segregated Witness~\cite{segwit} support, out of conservatism, or unawareness of such an upgrade.
\end{itemize}

In this work, we put forward a novel idea of combining VRF and mining in PoW to eliminate the intentionality of pooled mining, and to conquer the centralisation of mining power. We call our constrcution \textbf{VRF-based mining}.

\section{Preliminaries}

\subsection{Cryptocurrency mining}

\RH{describe how PoW and mining works}

% bitcoin whitepaper

% sok paper

\subsection{Verifiable random functions}

\TODO{diff between VRF \& VDF}

Verifiable Random Function (VRF)~\cite{} is a public-key version of cryptographic hash function.
It involves a pair of a public key and a secret key.
Only the owner of the secret key can compute the hash, while anyone having the public key can verify the hash.
Formally, a VRF consists of the following four algorithms: $\mathsf{VRFKeyGen}$, $\mathsf{VRFHash}$, $\mathsf{VRFProve}$ and $\mathsf{VRFVerify}$.

\begin{itemize}
    \item $(sk, pk) \gets \mathsf{VRFKeyGen}(1^{\lambda})$: on input a security parameter $1^{\lambda}$, outputs the private/public key pair $(sk, pk)$.
    \item $\beta \gets \mathsf{VRFHash}(sk, \alpha)$: on input $sk$ and an arbitrary-length string $\alpha$, outputs a hash $\beta$.
    \item $\pi \gets \mathsf{VRFProve}(sk, \alpha)$: on input $sk$ and $\alpha$, outputs the proof $\pi$ for $\beta$.
    \item $\{0, 1\} \gets \mathsf{VRFVerify}(pk, \alpha, \beta, \pi)$: on input $pk$, $\alpha$, $\beta$, $\pi$, outputs the verification result 0 or 1.
\end{itemize}
\section{VRF-based mining}

\subsection{Our construction}

As the digital signature scheme use in cryptocurrency are based on Elliptic curve, we construct Elliptic curve based VRF for our VRF-based mining.
The standardised construction is put in the appendix~\ref{}.

% word for construction (for example, follow mpc-sok)

\RH{replace hash by VRF...}

As shown in appendix~\ref{}, 


% what we need:
% unpredictability
% verifiability
% VRF fits in

\subsection{Why mining pools cannot work here}

% need private key

% no client would like to provide private key

% balance be spent
\section{Non-outsourceability analysis}
\label{sec:non_outsourceability}

% \cite{miller2015nonoutsourceable} first formally defines cryptocurrency mining as scratch-off puzzles. \TODO{describe}

% \begin{description}
%     \item[Correctness] Any valid hash can be verified.
%     \item[Feasibility] Mining is computationally feasible.
%     \item[Parallelisability] The mining process can be parallelised.
%     \item[$\mu$-Imcompressibility] The adversary cannot accelerate the mining process using pre-generated pairs of nonces and hashes, by at most a factor $\mu$ ($\mu \geq 1$). If $\mu = 1$, $\mathsf{Work}$ is the optimal way to mine.
%     \item[Non-transferability] Given a block template, the adversary cannot accelerate the mining process using other valid hashes on this block template.
% \end{description}

\subsection{Revised definitions}

Miller et al. \cite{miller2015nonoutsourceable} first formalises cryptocurrency mining as non-outsourceable scratch-off puzzles, and formally defines non-outsourceability.
In particular, they define two levels of non-outsourceability, namely \textbf{Weak Non-outsourceability} and \textbf{Strong Non-outsourceability}.

\begin{description}
    \item[Weak Non-outsourcability] If the pool operator outsources the mining process, miners can always steal the reward of mining.
    \item[Strong Non-outsourcability] In addition to the Weak Non-outsourcability, the pool operator cannot link the stolen mining reward with the miner who steals it.
\end{description}

Basically, \textbf{Weak Non-outsourceability} defines the punishment of outsourcing, while \textbf{Strong Non-outsourceability} covers both the punishment and the anonymity of malicious miners.
We call the property defining the punishment of outsourcing \textbf{Punish-mining-reward}.
The anonymity of malicious miners defined in~\cite{miller2015nonoutsourceable} is equivalent to \textbf{Transaction Unlinkability}~\cite{van2013cryptonote}: given two arbitrary transactions, distinguishing whether they have the same output address requires knowing the private keys holding their outputs.
To steal the mining reward, the miner should create a transaction, of which the input is the mining reward and the output is his owned address.
Identifying who steals the mining reward is equivalent to finding out who holds the output address, which can be further reduced to \textbf{Transaction Unlinkability}.


\subsection{Punish-secret-key-leakage}

% SK-non-outsourceability
We introduce \textbf{Punish-secret-key-leakage}, a new property of punishment on outsourcing that achieves stronger non-outsourceability than \textbf{Punish-mining-reward}.
A cryptocurrency mining protocol achieves \textbf{Punish-secret-key-leakage} in the following sense: the pool operator should reveal 1) the block template and 2) the private key receiving the coinbase transaction, so that miners can mine in the name of the pool operator.

% stronger than weak non-out
In a cryptocurrency mining protocol with \textbf{Punish-secret-key-leakage}, the pool operator should reveal his private key to miners, which gives miners opportunity to steal all cryptocurrency in the pool operator's wallet.
This indicates that, in terms of the punishment of outsourcing, \textbf{Punish-secret-key-leakage} is stronger than \textbf{Punish-mining-reward}, where miners can only steal the mining reward part.






\subsection{Non-outsourceability of VRF-based mining}

VRF-based mining achieves both \textbf{Punish-secret-key-leakage} and \textbf{Transaction Unlinkability}.
In VRF-based mining, the pool operator outsources mining requires revealing his private key to miners, which leads to \textbf{Punish-secret-key-leakage}.
Similar with the construction achieving \textbf{Strong Non-outsourceability} in~\cite{miller2015nonoutsourceable}, VRF-based mining achieves \textbf{Transaction Unlinkability} by allowing a malicious miner to use a freshly generated address to steal cryptocurrency.
To receive the stolen cryptocurrency of the pool operator, the malicious miner can create a new address, and construct a transaction of which the stolen cryptocurrency directs to this address.
As this new address has no historical transactions, linking the transaction stealing cryptocurrency with other transactions can be impossible.
Then the malicious miner can then spend his stolen cryptocurrency anonymously using numerous techniques, such as mixing services~\cite{maxwell2013coinjoin}\cite{bonneau2014mixcoin}\cite{ruffing2014coinshuffle}\cite{heilman2017tumblebit}, stealth addresses~\cite{van2013cryptonote}, and Zero Knowledge Contingent Payments~\cite{maxwell2016zero}.




\section{Instantiating VRF}

In order to implement VRF-based mining, one needs to first instantiate the VRF.
VRF has four configurable components, namely the elliptic curve and three hash functions $H_{1}(\cdot)$, $H_{2}(\cdot)$ and $H_{2}(\cdot)$.
$H_{1}(\cdot)$ maps an arbitrary-length string into an elliptic curve element;
$H_{2}(\cdot)$ maps an elliptic curve element into a fixed-length string; and
$H_{3}(\cdot)$ maps an arbitrary-length string into a fixed-length string.
In this section, we discuss considerations on choosing these four components for VRF-based mining.





\subsection{Elliptic curve}

As neither blockchains and VRF limits the choice of elliptic curves, any elliptic curve can be adapted.
Among prominent elliptic curves, Curve25519~\cite{} can be a promising choice.
Curve25519 supports Ed25519~\cite{}, a fast and secure digital signature algorithm.
In addition, numerous blockchains~\cite{} and projects using VRF~\cite{} adapt Ed25519 as their underlying elliptic curve.




\subsection{$H_{1}(\cdot)$ (hashing a string to an elliptic curve point)}

% indistinguishable
$H_{1}(\cdot)$ is a hash-to-curve function, which should prevent distinguishing behaviours: adversaries cannot learn any pattern of the input from its hash.
% deterministic
In addition, the hash-to-curve function used in VRF should be deterministic, otherwise the hash will be unreproducible.

A standardisation document~\cite{} specifies several hash-to-curve functions that fit into different elliptic curves and satisfy our requirements: Icart Method~\cite{}, Shallue-Woestijne-Ulas Method~\cite{}, Simplified SWU Method~\cite{} and Elligator2~\cite{}.
Within these hash-to-curve functions, Elligator2 is the recommended one for Curve25519.




\subsection{$H_{2}(\cdot)$ (hashing an elliptic curve point to a string)}

$H_{2}(\cdot)$ hashes an elliptic curve point to a fixed-length string.
It can be divided to two steps: 1) encoding an elliptic curve point to a string, and 2) hashing the string using a normal hash function.
The encoding step can be bidirectional and unencrypted, so can be done simply by converting a big integer to a string.
The hashing step should be cryptographically secure, so can adapt any existing cryptographically secure hash functions.

For ASIC-resistant cryptocurrencies using memory-hard hash functions (e.g., Ethereum~\cite{} and Monero~\cite{}), there are some concerns on choosing hash functions in $H_{2}(\cdot)$.
% VRFHash shoule memory-hard
To make mining memory-hard, $\mathsf{VRFHash}$ should be memory-hard.
% vrfhash steps
$\mathsf{VRFHash}$ of the standardised VRF consists of one $H_{1}(\cdot)$ hashing, one scalar-point multiplication and one $H_{2}(\cdot)$ hashing.
$H_{1}(\cdot)$ and the encoding step in $H_{2}(\cdot)$ can be fast so less possible to become the performance hotspot.
The scalar-point multiplication is slow and computation (rather than memory)-intensive.
% make VRFHash memory hard
The hash function in $H_{2}(\cdot)$ is the only possibility to make $\mathsf{VRFHash}$ memory-hard, and it should be overwhelmingly slower than the scalar-point multiplication, otherwise the scalar-point multiplication will make $\mathsf{VRFHash}$ computation-intensive.
This requirement can be achieved by repetitively executing one or multiple different memory-hard hash functions (such as Ethash~\cite{}, Equihash~\cite{}, and CryptoNight~\cite{}).





\subsection{$H_{3}(\cdot)$ (Normal hash function)}

$H_{3}(\cdot)$ is only used in $\mathsf{VRFProve}$ (proving the authorship of hashes) and $\mathsf{VRFVerify}$ (verifying the authorship of hashes).
The overhead of proving and verification should be minimised, so cryptographically secure hash functions that are designed to be fast (such as Keccak~\cite{} and BLAKE~\cite{}) are suitable for $H_{3}(\cdot)$.
\section{Possible problems and solutions}
\label{sec:problems-and-solutions}

While eliminating mining pools can contribute to decentralisation, it will also introduce new problems.
In this section, we discuss potential problems introduced by VRF-based mining, as well as how to address them.
In particular, we focus on two problems, namely the high reward variance and the high orphan rate.

\subsection{High reward variance}

As aforementioned, mining is probabilistic, and miners may not obtain stable income via solo mining, leading to the need of pooled mining.
In other words, unstable income may discourage miners to mine.
With weaker incentive of mining, the blockchain will have less mining power, which weakens the security of PoW-based consensus.
In particular, adversaries can make use of external mining power to compromise blockchains with less mining power~\cite{hansucker}.

A promising approach to address this problem is fine-graining the mining reward scheme.
% multi-tier
Miller et al.~\cite{miller2015nonoutsourceable} proposes the Multi-tier reward scheme, where the mining difficulty is divided into different levels, and miners can mine blocks satisfying different difficulty levels arbitrarily.
The multi-tier reward scheme distributes the mining reward in a fine-grained way so lowers the reward variance.
% strongchain
StrongChain~\cite{szalachowski2019strongchain} introduces the notion of Collaborative PoW, where miners are incentivised to mine blocks together and the mining reward is distributed in proportion to miners' contributions.
% prism
Prism~\cite{bagaria2019prism} decouples the block to core blocks and transaction blocks.
Core blocks construct the blockchain and only contain metadata, while transaction blocks are anchored on core blocks and contain confirmed transactions.
Mining core blocks and transaction blocks is different in terms of difficulty, and miners can mine core blocks and transaction blocks simultaneously.
In this way, miners are also rewarded more stably.

Another approach is to increase the speed of mining blocks.
With more blocks mined in a time unit, the mining reward can also be more stable.
For example, proposals on blockchain scalability such as sharding~\cite{wang2019monoxide} and DAG~\cite{li2018scaling} can stabilise the mining reward variance, although they are not designed for it.





\subsection{High orphan block rate}

Another problem is the high orphan block rate.
Due to the network latency, miners may mine blocks on the same height, but the blockchain will only accept one block and discard the rest blocks at last.
Such discarded blocks are called orphan blocks, and orphan blocks cannot obtain any mining reward.
As miners cannot mine together with VRF-mining, there will be more miners in the blockchain, leading to more orphan blocks.
With higher orphan block rate, miners will waste more mining power and obtain less mining reward.
Similar to high reward variance, miners will be discouraged to mine with high orphan blocks, which weakens the security of PoW-based consensus.

A possible approach that has already been deployed is rewarding orphan blocks instead of discarding them.
Ethereum~\cite{wood2014ethereum} adapts the GHOST protocol and introduces uncle blocks.
In Ethereum, orphan blocks can be included in the blockchain as uncle blocks, and miners finding uncle blocks will obtain mining reward, but less than normal blocks.

Another approach is to accelerate the block propagation to let miners know the latest block as soon as possible.
BIP-152~\cite{corallo2016bip} introduces the block relay protocol.
In the block relay protocol, miners synchronise their transaction pools in a real-time manner.
Once finding a block, the miner only broadcasts the compact version of it.
The compact block includes the block header and hashes of transactions in it, but does not include the transaction content.
Upon receiving a compact block, the miner reconstructs the block by matching transactions in his transaction pool with transaction hashes.
As the compact block is much smaller than the full block, broadcasting compact blocks can be much faster, and miners will be less possible to mine on earlier blocks.
Following BIP-152, several proposals on accelerating the block propagation have been proposed~\cite{ozisik2019graphene}\cite{klarman2018bloxroute}\cite{naumenko2019bandwidth}.


% \subsection{Why consensus requires mining}

% Nakamoto consensus requires PoW minig because it is designed that the consensus blockchain is the chain with the greatest PoW effort accumulated and therefore the greatest the greatest difficulty to produce, usually the longest chain.
% Such a chain is chosen as the majority agreement because it is the most difficult one to manipulate.

% Moreover, PoW can provide the blockchain with Sybil-resistance.
% Sybil attack is to subvert a reputation system by forging indentities.
% In PoW-style blockchains, the miner mines the new block is identically the leader producing the new block.
% That is to say, mining is to create the identity to be elected as the leader.
% As PoW requires computations, such a identity creation is neither costless.
% Therefore, PoW-style blockchains resist Sybil attacks by making identities generation expensive.

% \subsection{Why miners join mining}

% PoW-style blockchains incentivise miners to maintain the transaction history.

% Block subsidy and transaction fees are to encourage miners to keep recording new transaction.
% This help provide the blockchain with liveness: new blocks and valid transactions with appropriate feed will continue to be added to the ledger.

% The incentive also motivate the miners to stay honest and is to achieve incentive compatibility~\cite{}.
% Because the agreed chain has the corresponding PoW effort accumulated, to alter the history and double-spend his money, a miner needs to control a considerable amount of hash power to redo the PoW of the block, in which his first transaction was included, and all the following blocks since it.
% A miner owning such a computing power may choose between using it to double spend his money, or using it to mine new blocks and get rewarded. 

% To keep the blockchain system running expectedly, the reward needs to be profitable.
% Therefore, miners are incentivised to join mining.

% \subsection{Why miners join mining pools}

% As aforementioned, unsteady income stream will be risky, but the probability of a miner find a solution is in propotion to his hash power.
% Therefore, in pratice, miners often join a mining pool and perform computation on behalf of the pool, and get lower variant payouts allocated to them by the pool operators.

% This in fact relies on the fact that miners cannot steal the reward paying to the pool, because he neither can modify the coinbase transaction nor have knowledge on spending such an output;
% and the assumption that the pool will honestly pay to the miners according to their contribuitons.
% If a dishonest pool pays less to a miner than it deserves, the miner can discover it, leave the pool and shift to another.
% This will descrease the pool's mining power and lower his probability to mine a block, and no rational pool will do so.

% Therefore, the pool operator can trustlessly delegate the tasks to miners and miners tend to join pooled mining.
% However, this leads to the centralisation of computation power.

% \subsection{Balancing the sybil resistance and incentive}
% \label{sebsec:balancing}

% We have addressed in Section~\ref{sec:discourage-pool} that, our sheme completely discourage a rational pool operator who wants get mining reward.
% However, there still remains a concern that, we need to make a trade-off between Sybil resistance and incentivising miners to join mining.

% \HY{I find it difficult to connect it with Quadratic Voting or with Radical Market, so I skip it.}

% Assumes we have 2 different mining algorithms A and B, of which the computation-power-vs-investment-on-hardware curves are shown in Figure~\ref{fig:algo_A} and Figure~\ref{fig:algo_B} respectively.

% For algorithm A, with fewever investment on hardware, a miner can gain relativly higher computing power than B. That is, it is cheaper to generate identity, and hence it is less sybil-resistant.

% However, if using algorithm B, though it can bring stronger sybil resistance to the system, a miner will find that he cannot gain significant hash power unless he invest enough on the hardware.
% As aforementioned in Section~\ref{sec:intro}, low minig power will lead to unstable payouts, which will undoubtedly disincentivize miners if their budget are limited.

% Therefore, when designing the VRF, we need to take into account balancing the sybil resistance and incentive.

% \begin{figure}
% \centering
% \begin{tikzpicture}
%   \draw[->] (0,0) -- (6,0) node[right] {{Investment on Hardware}};
%   \draw[->] (0,0) -- (0,5) node[above] {{Hash Power}};
%   \draw (0,0) .. controls (3,0) and (4,0) .. (5,4.5);
% \end{tikzpicture}
% \caption{Algorithm A}
% \label{fig:algo_A}
% \end{figure}


% \begin{figure}
% \centering
% \begin{tikzpicture}
%   \draw[->] (0,0) -- (6,0) node[right] {{Investment on Hardware}};
%   \draw[->] (0,0) -- (0,5) node[above] {{Hash Power}};
%   \draw (0,0) .. controls (0,3) and (0,4) .. (5,4.5);
% \end{tikzpicture}
% \caption{Algorithm B}
% \label{fig:algo_B}
% \end{figure}

\section{Related work}
\label{sec:related}

To the best of our knowledge, VRF-based mining is the first construction that makes pooled mining \textit{impossible}.
In this section, we briefly review related research on preventing mining pools, and compare them with VRF-based mining.
We classify related research to two types, namely mining protocols trying to address pooled mining, and decentralised mining pools.

\subsection{Mining protocols}

There are two mining protocols aiming at discouraging or breaking mining pools: the \textit{non-outsourceable scratch-off puzzle}~\cite{miller2015nonoutsourceable} and the \textit{Two Phase Proof-of-Work} (\textit{2P-PoW})~\cite{2P-PoW}.

\textbf{Non-outsourceable scratch-off puzzle.}
Miller et al.~\cite{miller2015nonoutsourceable} formalises cryptocurrency mining as \textit{Scratch-off puzzles}, defines \textit{Non-outsourceable scratch-off puzzles}, and proposes two constructions.
One of these two constructions achieves \textbf{Weak non-outsourceability} (i.e., miners can steal the mining reward), and the other achieves \textbf{Strong non-outsourceability} (i.e., miners can anonymously steal the mining reward).

In the weak non-outsourceable scratch-off puzzle, mining consists of three steps.
First, the miner creates a Merkle tree randomly from the nonce.
Second, the miner produces a hash from the nonce and the Merkle tree.
If the hash meets the difficulty requirement, then go to the last step.
Last, the miner binds the nonce and his block template together to produce a valid block.
In order to outsource the mining process, the mining pool should distribute the search space of nonces to miners.
% how to steal
Miners can steal the mining reward by repetitively searching for a valid nonce (the first two steps) then binds this valid nonce with his own block template (the last step).
% why weak
However, in this protocol, all miners with the same view of the blockchain share the same search space of nonces, as the search space of nonces only relies on the previous block hash, rather than both the previous block hash and the block template like in Bitcoin.
By exploiting this fact, the pool operator can identify the miner who steals the mining reward.
For example, the pool operator can link the nonce in the stolen block with the miner who takes charge of the search space covering this nonce.
To achieve the \textbf{Strong non-outsourceability} (i.e., make the stealing behaviours anonymous), the strong non-outsourceable scratch-off puzzle replaces the plaintext nonce in the block with a Zero Knowledge Proof proving the statement ``I know a valid nonce''.

As discussed in \S\ref{sec:non_outsourceability}, compared to the non-outsourceable scratch-off puzzle, our VRF-based mining achieves better non-outsourceability, and is much simpler to implement.

\textbf{2P-PoW.}
Eyal and Sirer proposes 2P-PoW~\cite{2P-PoW}, a mining protocol that discourages pooled mining.
In 2P-PoW, there are two phases, and each phase has a difficulty parameter.
A miner should find a nonce that makes the block to pass two phases: 1) the Sha256d hash of the block meets the first difficulty, 2) the Sha256 hash of the signature of the block meets the second difficulty.
As the second requirement requires the private key, pool operators cannot outsource the second phase.

Compared to VRF-based mining that makes pooled mining impossible, 2P-PoW only discourages pooled mining, as the first phase is outsourceable.
In addition, 2P-PoW should use deterministic digital signatures, while commonly used digital signatures (e.g., ECDSA, EdDSA) rely on randomisation.
If the signature is non-deterministic, the pool operator can make use of all nonces that are generated by miners and meet the first difficulty.
For example, given a nonce meeting the first difficulty, the pool operator repetitively generates signatures to meet the second requirement.
Moreover, how to adjust two difficulties still remains unknown and requires some simulations.




\subsection{Decentralised mining pools}

% ref: https://www.alexeizamyatin.me/files/Decentralized_Mining-Security_and_Attacks.pdf


Decentralised mining pool is a type of mining pool that works in a decentralised way.
More specifically, miners mine in the name of themselves rather than the pool operator, but they share reward in a fine-grained way.
In this way, miners are rewarded stably while mining power is not aggregated to pool operators.



\textbf{P2Pool}~\cite{voight2011p2pool} is a decentralised mining pool for Bitcoin.
% how p2pool works
P2Pool runs a share-chain among all miners in the pool, and the share-chain includes shares submitted to the pool in sequence.
During mining, the coinbase transaction records the number of shares submitted by each miner, and distributes the mining reward according to miners' contribution.
In this way, once mining a Bitcoin block, the coinbase transaction will become valid, and miners will be rewarded according to their contribution.
% challenges
However, P2Pool suffers from several challenges.
First, handling the difficulty of mining shares in P2Pool is hard.
If the difficulty is high, a miner's reward will still be volatile.
If the difficulty is low, there will be numerous low-difficulty shares, which introduces huge overhead on broadcasting shares or even spamming attacks.
Second, frequent share submissions amplifies the influence of network latency which leads to high orphan share rate.
Last, P2Pool is vulnerable to temporary dishonest majority~\cite{decentralised-mining-pool-security}.

\textbf{SmartPool}~\cite{luu2017smartpool} is another decentralised mining pool, which uses a smart contract to replace the centralised pool operator.
% con 1: smart contract
As relying on smart contracts, SmartPool cannot work on blockchains without smart contracts.
% con 2: network delay
In addition, as blockchains achieve limited throughput, transaction processing might be congested and mining reward might be delayed, especially when a large number of miners participate in the SmartPool.
% con 3: compute and storage overhead
Moreover, as the SmartPool smart contract should verify the validity of blocks, miners should submit the whole block (with transactions) to the SmartPool.
Verifying blocks introduces significant overhead on computing (so expensive transaction fees), and storing blocks in the SmartPool smart contract also introduces significant overhead on storing the blockchain.



\textbf{BetterHash}~\cite{draft-bip-BetterHash} is another decentralised mining protocol, which has been integrated into \textbf{Stratum V2}~\cite{stratum-v2}, the next generation of the \textit{Stratum}~\cite{stratum} pooled mining protocol.
In \textit{BetterHash}, the block operator allows miners to choose transactions and construct blocks in his name, rather than constructing block templates by himself.
Thus, \textit{BetterHash} only contributes to the decentralisation of constructing blocks, but does not contribute to the decentralisation of mining power.

% In \textit{Stratum V2}, a pool will need to send the block template to the miners,
% and a miner may additionally communication with a bitcoin node, handle transaction selection to build the block it expects, and negotiate with the pool for one or multiple time,
% which will undoubtedly increase the communication overhead and operation complexity.
% Moreover, \textit{Stratum V2} leaves \textit{BetterHash} optional but not mandatory, which implies that it make no promise to the decentralisation.
% And, if miners have divergences, as the negotiation may divide pool's minig power among its miners and introduce inefficiency, 
% a pool may not be incentivised to provide such a functionality.

\section{Conclusion and future work}
\label{sec:conclusion}

In this paper, we propose VRF-based mining, that can make pooled mining in Proof-of-work-based consensus impossible.
VRF-based mining is simple and intuitive: miners produce hashes of blocks using VRFs rather than hash functions, so a pool operator should reveal his private key to outsource the mining process to other miners.
In addition, we informally show the non-outsourceability of VRF-based mining and describe how to instantiate VRFs for implementing VRF-based mining.
Moreover, we discuss potential problems of having no mining pools and how to address them.

While formally defining non-outsourceable cryptocurrency mining and VRF-based mining takes more research effort, we would leave it to future work.
Also, as aforementioned, ruling out pooled mining can achieve better decentralisation but may harm the incentive of mining.
How to achieve decentralisation while preserving the incentive of mining is still an open challenge.

\bibliographystyle{splncs04}
\bibliography{refs}

\appendix
\section{Detailed experimental results}

\begin{table}
    \centering
    \begin{tabular}{|l|l|l|l|l|l|}
        \hline
        Coin & Region & Host                      & Port  & Mean latency & Std. of latency \\
        \hline
        btc  & us     & btc-us.f2pool.com         & 3333  & 228.845       & 20.548227    \\
        btc  & asia   & btc.ss.poolin.com         & 443   & 165.696       & 31.206496    \\
        btc  & eu     & btc-eu.f2pool.com         & 3333  & 162.293       & 15.072790    \\
        btc  & eu     & eu.ss.btc.com             & 1800  & 158.438       & 4.402037     \\
        btc  & eu     & eu1.btc.sigmapool.com     & 3333  & 214.876       & 29.206920    \\
        eth  & us     & eth-us-east1.nanopool.org & 9999  & 226.004       & 1.173354     \\
        eth  & us     & eth-us-west1.nanopool.org & 9999  & 182.209       & 5.008756     \\
        eth  & asia   & eth-jp1.nanopool.org      & 9999  & 260.082       & 34.864215    \\
        eth  & eu     & eth-eu1.nanopool.org      & 9999  & 284.720       & 37.420361    \\
        eth  & eu     & eth-eu2.nanopool.org      & 9999  & 291.281       & 6.972029     \\
        eth  & eu     & eu-eth.hiveon.net         & 4444  & 247.869       & 19.166674    \\
        eth  & eu     & eth-eu.dwarfpool.com      & 80    & 306.198       & 29.074232    \\
        ltc  & us     & us.litecoinpool.org       & 3333  & 231.411       & 18.369690    \\
        ltc  & us     & us2.litecoinpool.org      & 3333  & 167.318       & 3.225467     \\
        ltc  & us     & us-ltc.ss.btc.com         & 1800  & 225.951       & 11.944215    \\
        ltc  & us     & ltc-us.f2pool.com         & 8888  & 230.549       & 25.844133    \\
        xmr  & us     & xmr-us-east1.nanopool.org & 14444 & 226.526       & 2.084073     \\
        xmr  & us     & xmr-us-west1.nanopool.org & 14444 & 187.440       & 15.914913    \\
        xmr  & us     & monerohash.com            & 2222  & 224.457       & 24.607601    \\
        xmr  & asia   & xmr-jp1.nanopool.org      & 14444 & 235.547       & 9.082764     \\
        xmr  & asia   & gulf.moneroocean.stream   & 80    & 228.405       & 8.530898     \\
        xmr  & eu     & xmr-eu1.nanopool.org      & 14444 & 241.316       & 24.772531    \\
        xmr  & eu     & xmr-eu2.nanopool.org      & 14444 & 238.899       & 24.653803    \\
        xmr  & eu     & xmrpool.eu                & 3333  & 248.898       & 7.124086     \\
        zec  & us     & zec.slushpool.com         & 4444  & 222.598       & 7.565555     \\
        zec  & eu     & zec-eu.luxor.tech         & 6666  & 295.176       & 4.036088     \\
        \hline
    \end{tabular}
\end{table}

\end{document}